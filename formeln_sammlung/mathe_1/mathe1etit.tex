% Analysis 1 Formelsammlung
% B.Sc. Elektrotechnik und Informationstechnik, TUM, 1. Semester

% Dokumenteinstellungen
% ======================================================================

% Dokumentklasse (Schriftgröße 6, DIN A4, Artikel)
\documentclass[6pt,a4paper]{scrartcl}

% Pakete laden
\usepackage[utf8]{inputenc}		% Zeichenkodierung: UTF-8 (für Umlaute)
\usepackage[german]{babel}		% Deutsche Sprache
\usepackage{multicol}			% Spaltenpaket
\usepackage{amsmath}
\usepackage{amssymb}
\usepackage{esint}				% erweiterte Integralsymbole
\usepackage{multicol}			% ermöglicht Seitenspalten
\usepackage{wasysym}			% Blitz
\usepackage{graphicx}

% Seitenlayout und Ränder:
\usepackage{geometry}
\geometry{a4paper,landscape, left=6mm,right=6mm, top=0mm, bottom=3mm,includeheadfoot}

%Kopf- und Fußzeile
\usepackage{fancyhdr}
\pagestyle{fancy}
\fancyhf{}

   %\fancyfoot[C]{\textbf{Analysis 1} von Lukas Kompatscher (lukas.kompatscher@tum.de)}
   \renewcommand{\headrulewidth}{0.0pt} %obere Linie ausblenden
   \renewcommand{\footrulewidth}{0.1pt} %untere Linie

   \fancyfoot[R]{Stand: \today \qquad \thepage}
   %\fancyfoot[L]{Homepage: www.latex4ei.de - Fehler bitte sofort melden.}

% Schriftart SANS für bessere Lesbarkeit bei kleiner Schrift
\renewcommand{\familydefault}{\sfdefault}


% Custom Commands
\renewcommand{\thesubsection}{\arabic{subsection}}
\newcommand{\me}[1]{\ensuremath{\left\{#1\right\}}}
\newcommand{\dme}[2]{\ensuremath{\left\{#1\,\vert\,#2 \right\}}}
\newcommand{\abs}[1]{\ensuremath{\left\vert#1\right\vert}}
\newcommand{\un}[1]{\; \unit{#1} }
\newcommand{\unf}[2]{\;\left[ \unitfrac{#1}{#2} \right]}
\newcommand{\norm}[2][\relax]{\ifx#1\relax \ensuremath{\left\Vert#2\right\Vert}\else \ensuremath{\left\Vert#2\right\Vert_{#1}}\fi}
\newcommand{\enbrace}[1]{\ensuremath{\left(#1\right)}}
\newcommand{\nira}[1]{\ensuremath{\overset{n \rightarrow \infty}{\longrightarrow}}}
\newcommand{\os}[2]{\ensuremath{\overset{#1}{#2}}}
\makeatletter
\newcommand{\Ra}[0]{\ensuremath{\Rightarrow}}
\newcommand{\ra}[0]{\ensuremath{\rightarrow}}
\newcommand{\gk}[1]{\ensuremath{\left\lfloor#1\right\rfloor}}
\newcommand{\sprod}[2]{\ensuremath{%
  \setbox0=\hbox{\ensuremath{#2}}
  \dimen@\ht0
  \advance\dimen@ by \dp0
  \left\langle #1\rule[-\dp0]{0pt}{\dimen@},#2\right\rangle}}

%Custom functions
\DeclareMathOperator{\arccot}{arccot}

% Dokumentbeginn
% ======================================================================
\begin{document}
%\section{}
% ----------------------------------------------------------------------

% Aufteilung in Spalten
\begin{multicols*}{4}
\parbox{2.3cm}{ Mathematik 1
}
\parbox{4cm}{
	\huge{\textbf{Mathematik 1}}
}
\subsection{Allgemeines} % (fold)
\label{sub:allgemeines}

Dreiecksungleichung \qquad \qquad \qquad
\begin{math}\begin{array}{l}
	\abs{x + y} \le \abs{x} + \abs{y} \\
	\abs{\abs{x}- \abs{y}} \le \abs{x-y}
\end{array}\end{math} \\
Cauchy-Schwarz-Ungleichung: \qquad
\begin{math}\begin{array}{l}
\left| \sprod{x}{y} \right| \le \| x\| \cdot \| y\|
\end{array}\end{math} \\
\\
Arithmetische Summenformel \qquad
\begin{math}\begin{array}{l}
	\sum\limits_{k = 1}^{n}k = \frac{n (n+1)}{2}
\end{array}\end{math}  \\
\\
Geometrische Summenformel \qquad
\begin{math}\begin{array}{l}
	\sum\limits_{k = 0}^{n}q^k = \frac{1 - q^{n+1}}{1-q}
\end{array}\end{math}\\
\\
Bernoulli-Ungleichung \qquad \qquad \quad
\begin{math}\begin{array}{l}
	(1+a)^n \ge 1 + na
\end{array}\end{math}\\
\\
Binomialkoeffizient \qquad \qquad \qquad
\begin{math}\begin{array}{l}
	\binom{n}{k} = \frac{n!}{k!(n-k)!}  \\
	\binom{n}{0} = \binom{n}{n} = 1
\end{array}\end{math}\\
\\
Binomische Formel \qquad \qquad \qquad
\begin{math}\begin{array}{l}
	(a+b)^n = \sum\limits_{k = 0}^{n} \binom{n}{k} a^{n-k} b^{k}
\end{array}\end{math}   \\
\\
Äquivalenz von Masse und Energie
\begin{math}\begin{array}{l}
	E = mc^2
\end{array}\end{math}\\
\\
Wichtige Zahlen: $\sqrt{2} = 1,41421$\quad $\pi=$ ist genau 3 \quad $e = 2,71828$ \quad $\pi =  3,14159$

\paragraph{Fakultäten} % (fold)
\label{par:fakultaeten}
$n! = 1 \cdot 2 \cdot 3 \cdot \ldots \cdot n$ \qquad  $0! = 1! = 1$

% paragraph fakultäten (end)
% subsubsection subsection_name (end)
% subsection allgemeines (end)


\subsection{Komplexe Zahlen}
% ----------------------------------------------------------------------
Eine komplexe Zahl $z=a+b\mathbf{i},\ z\in \mathbb C, \quad a,b \in \mathbb R$ besteht aus einem Realteil $\Re(z)=a$ und einem Imaginärteil $\Im(z)=b$, wobei $\mathbf{i}=\sqrt{-1}$ die immaginären Einheit ist.
Es gilt: \quad $i^2 = -1$ \quad $i^4 = 1$
\subsubsection{Kartesische Koordinaten}
Rechenregeln:\\
$z_1+z_2=(a_1+a_2)+(b_1+b_2)\mathbf{i}$\\
$z_1\cdot z_2=(a_1\cdot a_2-b_1\cdot b_2)+(a_1\cdot b_2+a_2\cdot b_1)\mathbf{i}$\\
\\
Konjugiertes Element von $z=a+b\mathbf{i}$:\\
$\overline{z}=a-b\mathbf{i}$\qquad \qquad \qquad \qquad \qquad \qquad \qquad \qquad $e^{\overline{ix}} = e^{-ix}$  \\
$z\overline{z}=|z|^2=a^2+b^2$\\
\\
Inverses Element:\\
$z^{-1}=\frac{1}{z}\frac{\overline z}{\overline z z}=\frac{\overline z}{a^2+b^2}=\frac{a}{a^2+b^2} - \frac{b}{a^2+b^2}\mathbf{i}$


\subsubsection{Polarkoordinaten}
$z=a+b\mathbf{i}\ne0$\ in Polarkoordinaten:\\
$z=r (\cos(\varphi)+\mathbf{i}\sin(\varphi))=r\cdot e^{\mathbf{i} \varphi}$\\
$r=|z|=\sqrt{a^2+b^2}\quad\varphi=\arg(z)=\begin{cases}+\arccos \left( \frac{a}{r}\right),  & b\ge0   \\  -\arccos \left( \frac{a}{r}\right), & b<0\end{cases}$

\begin{description}\itemsep0pt
\item[Multiplikation:] $z_1\cdot z_2=r_1 \cdot r_2 ( \cos ( \varphi_1 + \varphi_2) + \mathbf{i} \sin (\varphi_1 + \varphi_2))$
\item[Division:] $\frac{z_1}{z_2}=\frac{r_1}{r_2} ( \cos ( \varphi_1 - \varphi_2) + \mathbf{i} \sin (\varphi_1 - \varphi_2))$
\item[n-te Potenz:] $z^n=r^n\cdot e^{n\varphi \mathbf{i}}= r^n (\cos (n \varphi) + \mathbf{i} \sin (n \varphi))$
\item[n-te Wurzel:] $\sqrt[n]{z}= z_k = \sqrt[n]{r} \left(\cos \left(\frac{\varphi + 2k\pi}{n}\right) + \mathbf{i} \sin \left(\frac{\varphi + 2k\pi}{n}\right)\right) \\ k =0,1, \ldots, n-1$
\item[Logarithmus:] $\ln(z)=\ln(r) + \mathbf{i}(\varphi + 2k\pi)$ \quad (Nicht eindeutig!)
\end{description}
Anmerkung: Addition in kartesische Koordinaten umrechnen(leichter)!

\subsection{Funktionen}
Eine Funktion $f$ ist eine Abbildung, die jedem Element $x$ einer Definitionsmenge $D$ genau ein Element $y$ einer Wertemenge $W$ zuordnet.\\
$f:D\rightarrow W,\ x \mapsto f(x):=y$\\
\\
\textbf{Injektiv}: $f(x_1)=f(x_2) \Rightarrow x_1=x_2$\\
\textbf{Surjektiv}: $\forall y\in W \exists x\in D:f(x)=y$\\ \quad (Alle Werte aus $W$ werden angenommen.)\\
\textbf{Bijektiv}(Eineindeutig): $f$ ist injektiv und surjektiv $\Rightarrow$ $f$ umkehrbar. \\
\textbf{Ableitung der Umkehrfunktion} \\
$f$ stetig, streng monoton, an $x_0$ diff'bar und $y_0=f(x_0) \\
\Rightarrow \enbrace{f^{-1}}(y_0)= \frac{1}{f'(x_0)} =\frac{1}{f'(f^{-1}\enbrace{y_0})}$

\subsubsection{Symmetrie einer Funktion $f$}
\textbf{Achsensymmetrie} (gerade Funktion): $f(-x)=f(x)$\\
\textbf{Punktsymmetrie} (ungerade Funktion): $f(-x)=-f(x)$\\
\\
Regeln für gerade Funktion $g$ und ungerade Funktion $u$:\\
$g_1 \pm g_2 = g_3$ \qquad $u_1 \pm u_2 = u_3$\\
$g_1 \cdot g_2=g_3$ \qquad $u_1 \cdot u_2 = g_3$ \qquad $u_1 \cdot g_1=u_3$

\subsubsection{Kurvendiskussion von $f: I = [a, b] \rightarrow \mathbb{R}$}
\textbf{Kandidaten für Extrama (lokal, global)}
\begin{enumerate}\itemsep0pt
\item Randpunkte von $I$
\item Punkte in denen $f$ nicht diffbar ist
\item Stationäre Punkte ($f'(x)=0$) aus $(a,b)$
\end{enumerate}
\textbf{Lokales Maximum} \\
wenn $x_0$ stationärer Punkt ($f'(x_0) = 0$) und
\begin{itemize}\itemsep0pt
\item
$f''(x_0)<0$ oder
\item
$f'(x) > 0, x \in (x_0-\varepsilon, x_0) \\
 f'(x) < 0, x \in (x_0, x_0+\varepsilon)$
\end{itemize}
\textbf{Lokales Minimum} \\
wenn $x_0$ stationärer Punkt ($f'(x_0) = 0$) und
\begin{itemize}\itemsep0pt
\item
$f''(x_0)>0$ oder
\item
$f'(x) < 0, x \in (x_0-\varepsilon, x_0) \\
 f'(x) > 0, x \in (x_0, x_0+\varepsilon)$
\end{itemize}
%$f'(x_0) = 0 \quad \begin{cases}f''(x_0)<0 \ \rightarrow \ \text{Maximum (lokal)} \\ f''(x_0)>0 \ \rightarrow \ \text{Minimum (lokal)} \end{cases} $\\
\textbf{Monotonie} \\
$f'(x) \underset{(>)}{^{\ge}} 0 \rightarrow$ \ $f$ (streng) Monoton steigend, $x\in(a,b)$\\
$f'(x) \underset{(<)}{^{\le}} 0 \rightarrow$ \ $f$ (streng) Monoton fallend, $x\in(a,b)$\\
\textbf{Konvex/Konkav} \\
$f''(x) \underset{(>)}{^{\ge}} 0 \ \rightarrow$ \ $f$ (strikt) konvex, $x\in(a,b)$\\
$f''(x) \underset{(<)}{^{\le}} 0 \ \rightarrow$ \ $f$ (strikt) konkav, $x\in(a,b)$\\
$f''(x_0)=0 \text{ und } f'''(x_0) \ne 0 \rightarrow x_0$ Wendepunkt \\
$f''(x_0)=0 \text{ und Vorzeichenwechseln an } x_0 \rightarrow x_0$ Wendepunkt \\
\subsubsection{Asymptoten von $f$}
Horizontal: $c=\lim\limits_{x\ra \pm \infty} f(x)$\\
Vertikal: $\exists \text{ Nullstelle } a \text{ des Nenners }: \lim\limits_{x \rightarrow a^{\pm}} f(x) = \pm \infty$\\
Polynomasymptote $P(x)$: $f(x):=\frac{A(x)}{Q(x)}=P(x)+ \underset{\ra 0}{\frac{B(x)}{Q(x)}}$ \\
\subsubsection{Wichtige Sätze für \underline{stetige} Fkt. $f: [a,b] \rightarrow \mathbb R, f \mapsto f(x)$ }
\textbf{Zwischenwertsatz:} $\forall y \in [f(a),f(b)]\ \exists x\in [a,b]:f(x)=y$\\
\textbf{Satz von Rolle:} Falls $f(a)=f(b)$, dann $\exists x_0: f' (x_0) = 0$\\
\textbf{Mittelwertsatz:} Falls $f$ diffbar, dann $\exists x_0:f'(x_0)=\frac{f(b)-f(a)}{b-a}$\\
\textbf{Regel von L'Hospital}:\\ $\lim\limits_{x \rightarrow a} \frac{f(x)}{g(x)} = \left[ \frac{0}{0} \right] / \left[ \frac{\infty}{\infty} \right] \rightarrow \lim\limits_{x \rightarrow a} \frac{f(x)}{g(x)} = \lim\limits_{x \rightarrow a} \frac{f'(x)}{g'(x)}$

% ----------------------------------------------------------------------
\subsubsection{Polynome $P(x)\in\mathbb R[x]_n$}
$P(x)=\sum_{i=0}^n a_ix^i=a_n x^n+a_{n-1} x^{n-1}+\dotsc+a_1x+a_0$ \\
Lösungen für $ax^2+bx+c=0$ \\
\begin{tabular}{l|l}
\textbf{Mitternachtsformel}:  &  Satz von Vieta:\\
$x_{1/2}=\frac{-b\pm\sqrt{b^2-4ac}}{2a}$  \quad & \quad   $x_1 + x_2 = - \frac{b}{a} \qquad x_1 x_2 = \frac{c}{a}$
\end{tabular}

\subsubsection{Trigonometrische Funktionen}
$f(t)=A\cdot \cos(\omega t + \varphi_0)=A\cdot \sin(\omega t + \frac{\pi}{2}+ \varphi_0)$
\begin{eqnarray*}
	\sin (-x) = -\sin (x)  \quad & \quad \cos (-x) = \cos (x) \\
	\sin^2 x + \cos^2 x = 1  \quad & \quad \tan x = \frac{\sin x}{\cos x} \\
	e^{ix}=\cos(x)+i\sin(x) \quad & \quad e^{-ix}=\cos(x)-i\sin(x) \\
	\sin(x)=\frac{1}{2i}\enbrace{e^{ix}-e^{-ix}} \quad & \quad \cos(x)=\frac{1}{2}\enbrace{e^{ix}+e^{-ix}} \\
	\sinh(x)=\frac{1}{2}(-e^{-x}+e^x) & \cosh(x)=\frac{1}{2}(e^{-x}+e^x)
\end{eqnarray*}
\paragraph{Additionstheoreme} % (fold)
\label{par:additionstheoreme}
 \begin{eqnarray*}
 	& \cos (x + y) = \cos x \cos y - \sin x \sin y \\
	& \cos \enbrace{x - \frac{\pi}{2}} = \sin x \qquad \quad \sin \enbrace{x + \frac{\pi}{2}} = \cos x \\
    & \sin \enbrace{x + y} = \sin x \cos y + \cos x \sin y \\
& 	\sin 2x = 2 \sin x \cos x        \\
	& \cos 2x = \cos^2 x - \sin^2 x = 2\cos^2 x - 1\\
 \end{eqnarray*}
% paragraph additionstheoreme (end)
\hspace{-20pt}
\scalebox{0.77}
{
$\begin{array}{c|c|c|c|c|c|c|c|c|c|c|c}
x & 0& 30 & 45& 60 & 90 & 120 & 135& 150& 180 & 270 & 360 \\ \hline
x & 0 & \pi / 6 & \pi / 4 & \pi / 3 & \pi / 2 & \frac{2}{3}\pi& \frac{3}{4}\pi& \frac{5}{6}\pi& \pi  & \frac{3}{2}\pi & 2 \pi \\ \hline
\sin & 0 & \frac{1}{2} & \frac{1}{\sqrt{2}} & \frac{\sqrt 3}{2} & 1 & \frac{\sqrt 3}{2} & \frac{1}{\sqrt{2}} & \frac{1}{2} & 0 & -1 & 0 \\
\cos & 1 & \frac{\sqrt 3}{2} & \frac{1}{\sqrt 2} & \frac{1}{2} & 0 & -\frac{1}{2} & -\frac{1}{\sqrt 2}  & -\frac{\sqrt 3}{2}   & -1 & 0 & 1 \\
\tan & 0 & \frac{\sqrt{3}}{3}&1 &\sqrt{3} & \lightning & -\sqrt{3}& -1& -\frac{1}{\sqrt{3}} & 0 & \lightning & 0\\
\end{array}$
}

\subsubsection{Potenzen/Logarithmus}
\begin{equation*}
\ln(u^r)=r\ln u
\end{equation*}

\subsection{Folgen}
% ----------------------------------------------------------------------
Eine Folge ist eine Abbildung $a: \mathbb N_0 \rightarrow \mathbb R,\ n \rightarrow a(n) =: a_n$\\
explizite Folge: $(a_n)$ mit $a_n=a(n)$\\ rekursive Folge: $(a_n)$ mit $a_0=f_0,\  a_{n+1}=a(a_n)$\\

\subsubsection{Monotonie}
Im Wesentlichen gibt es 3 Methoden zum Nachweis der Monotonie.\\
Für \textbf{(streng) monoton fallend} gilt:
\begin{enumerate}\itemsep0pt
\item $a_{n+1} - a_n \underset{(<)}{^{\le}} 0$
\item $\frac{a_n}{a_{n+1}} \underset{(>)}{^{\ge}} 1$ \qquad $\lor$ \qquad $\frac{a_{n+1}}{a_n} \underset{(<)}{^{\le}} 1$
\item Vollständige Induktion: $\forall n \in \mathbb{N}: a_{n+1}\underset{(<)}{^{\le}} a_n$
\end{enumerate}
\subsubsection{Konvergenz}
$(a_n)$ ist \emph{Konvergent} mit \emph{Grenzwert} $a$, falls: $\forall \epsilon > 0 \ \exists N  \in \mathbb N_0:  \abs{a_n -a} < \epsilon \ \forall n \ge N$\\
Eine Folge konvergiert gegen eine Zahl $a$:\ $(a_n) \overset{n \rightarrow \infty}{\longrightarrow} a$
\paragraph{Es gilt:}
\begin{itemize}\itemsep0pt
\item Der Grenzwert a einer Folge $(a_n)$ ist eindeutig.
\item Ist $(a_n)$ Konvergent, so ist $(a_n)$ beschränkt
\item Ist $(a_n)$ unbeschränkt, so ist $(a_n)$ divergent.
\item \emph{Das Monotoniekriterium}: Ist $(a_n)$ beschränkt und monoton, so konvergiert $(a_n)$
\item \emph{Das Cauchy-Kriterium:} Eine Folge $(a_n)$ konvergiert gerade dann, wenn: \\ $\forall \epsilon >0 \, \exists \,  N \in \mathbb N_0: \abs{a_n - a_m} < \epsilon \, \forall n, m \ge N$
\end{itemize}
Regeln für konvergente Folgen $(a_n) \overset{n \rightarrow \infty}{\longrightarrow} a$ und $(b_n) \overset{n \rightarrow \infty}{\longrightarrow} b$:\\
$\begin{array}{lll}
(a_n+b_n) \overset{n \rightarrow \infty}{\longrightarrow} a+b & (a_n b_n) \overset{n \rightarrow \infty}{\longrightarrow} ab & (\frac{a_n}{b_n}) \overset{n \rightarrow \infty}{\longrightarrow} \frac{a}{b}\\
(\lambda a_n) \overset{n \rightarrow \infty}{\longrightarrow} \lambda a & (\sqrt{a_n}) \overset{n \rightarrow \infty}{\longrightarrow} \sqrt{a} & (|a_n|) \overset{n \rightarrow \infty}{\longrightarrow} |a|
\end{array}$
\paragraph{Grenzwert bestimmen:}
\begin{itemize}\itemsep0pt
\item Wurzeln: Erweitern mit binomischer Formel
\item Brüche: Zähler und Nenner durch den Koeffizient höchsten Grades teilen
\item Rekursive Folgen: Fixpunkte berechnen. Fixpunkte sind mögliche Grenzwerte. Monotonie durch Vergleich $a_{n+1}$ und $a_n$ zeigen. Beschränktheit mit Induktion beweisen.
\end{itemize}

\subsubsection{Wichtige Regeln}
$a_n=q^n \quad \overset{n \rightarrow \infty}{\longrightarrow} \quad \begin{cases} 0 & |q|<1 \\ 1 & q=1 \\ \pm \infty & q < -1  \\  + \infty & q > 1\end{cases}$ \\
$a_n=\frac{1}{n^k}\rightarrow 0$ \qquad $\forall k \ge 1$\\
$a_n=\left(1+\frac{c}{n}\right)^n \rightarrow e^c$ \\
$a_n=n\left(c^{\frac1{n}}-1\right) = \ln c$\\
$a_n=\frac{n^2}{2^n}\ra 0$ \qquad \qquad \qquad ($2^n \ge n^2$ \quad $\forall n\ge 4$) \\
$\lim\limits_{n\to\infty}n^{\frac{1}{n}}=\lim\limits_{n\to\infty}\sqrt[n]{n}=1$


\subsubsection{Limes Inferior und Superior}
Der Limes superior einer Folge $x_n \subset \mathbb{R}$ ist der größte Grenzwert konvergenter Teilfolgen $x_{n_k}$ der Folge ${x_n}$ \\ \\
Der Limes inferior einer Folge $x_n \subset \mathbb{R}$ der kleinste Grenzwert konvergenter Teilfolgen $x_{n_k}$ der Folge $x_n$

\begin{sectionbox}

\end{sectionbox}
\subsection{Reihen}
% ----------------------------------------------------------------------
\begin{eqnarray*}
	\underset{\text{Harmonische Reihe}}{\sum_{n=1}^\infty \frac{1}{n} = \infty} \qquad \qquad  \qquad \qquad \underset{\text{Geometrische Reihe}}{\sum_{n=0}^\infty q^n = \frac{1}{1-q}} \qquad |q|<1
\end{eqnarray*}
\begin{eqnarray*}
	\sum_{n=1}^\infty \frac{1}{n^\alpha} = \begin{cases} \text{konvergent}, & \alpha > 1 \\ \text{divergent}, & \alpha \le 1 \end{cases} \qquad \qquad \qquad \qquad \qquad \qquad
\end{eqnarray*}



\subsubsection{Konvergenzkriterien}
$\sum^{\infty}_{n = 0} a_n$ divergiert, falls $a_n \not \rightarrow 0$ oder\\
Minorante:$\exists \sum^{\infty}_{n = 0} b_n (divergiert) \quad \land \quad a_n \ge b_n \quad \forall n\ge n_0$\\[0.6em]
$\sum^{\infty}_{n = 0}(-1)^n a_n$ konvergiert, if $(a_n)$ monoton fallende Nullfolge (Leibnitz)\\
oder Majorante: $\exists \sum^{\infty}_{n = 0} b_n = b \quad \land \quad a_n \le b_n \quad \forall n\ge n_0$\\
\\
Absolute Konvergenz($\sum^\infty_{n=0} |a_n|=a$ konvergiert), falls:\\
1. Majorante: $\exists \sum^{\infty}_{n = 0} b_n = b \quad \land \quad |a_n| \le b_n \quad \forall n\ge n_0$\\
2. Quotienten und Wurzelkriterium (BETRAG nicht vergessen!)
\begin{eqnarray*}
	\rho := \lim_{n \rightarrow \infty} \abs{\frac{a_{n+1}}{a_n}} \qquad \lor \qquad \rho := \lim_{n \rightarrow \infty} \sqrt[n]{\abs{a_n}} \qquad \forall n > N\\
	\text{Falls}
	\begin{cases}
		\rho < 1 \Ra  ~\sum^\infty_{n=0} a_n \text{ konvergiert absolut} \\
		\rho > 1 \Ra  ~\sum^\infty_{n=0} a_n \text{ divergiert} \\
		\rho = 1 \Ra  ~\sum^\infty_{n=0} a_n \text{ keine Aussage möglich}
	\end{cases}
\end{eqnarray*}
\\
Jede absolute konvergente Reihe ($\sum^\infty_{n=0} |a_n|$) ist konvergent ($\sum^\infty_{n=0} a_n$)


\subsection{Potenzreihen} % (fold)
% ----------------------------------------------------------------------
\begin{equation*}
f(x)=\sum_{n=0}^\infty a_n \cdot (x-c)^n
\end{equation*}
\subsubsection{Konvergenzradius}
$R = \lim\limits_{n\rightarrow \infty} \abs{\frac{a_n}{a_{n+1}}}=\frac{1}{\lim\limits_{n\rightarrow \infty}\sqrt[n]{\abs{a_n}}}$ \\
$R =\liminf\limits_{n\rightarrow \infty} \abs{\frac{a_n}{a_{n+1}}}=\frac{1}{\limsup\limits_{n\rightarrow \infty}\sqrt[n]{\abs{a_n}}}$ \\ \\
$f(x)\begin{cases}
	\text{konvergiert absolut} & \abs{x-c} < R \\
	\text{divergiert} & \abs{x-c} > R \\
	\text{keine Aussage möglich} & \abs{x-c} = R
	\end{cases}$\\ \\
Bei reellen Reihen gilt: \\
$\Ra x$ konvergiert im offenen Intervall $I=(c-R,c+R)$ \\
$\Ra$ Bei $x=c-R$ und $x=c+R$ muss die Konvergenz zusätzlich überprüft werden.\\\\
Substitution bei $f(x)=\sum_{n=0}^\infty a_n \cdot x^{\lambda n}$ \\
$w=x^\lambda \rightarrow x=w^\frac{1}{\lambda}$
$\rightarrow R=\left(R_w\right)^\frac{1}{\lambda}$

%-----------------------------------------
%Konvergenz:\\
%$\abs{\frac{a_{n+1} (x-a)^{n+1}}{a_n (x-a)^n}} = \abs{\frac{a_{n+1}}{a_n}}\abs{x-a} \overset{n \rightarrow \infty}{\rightarrow} q \cdot \abs{x -a}$\\
%Konvergenzradius: $R=\frac{1}{q}$\\
%-----------------------------------------
\subsubsection{Wichtige Potenzreihen}
\label{sub:potenzreihen}
\begin{equation*}
 	e^x = \sum_{n = 0}^{\infty} \frac{x^n}{n!} = \lim\limits_{n\to\infty}\enbrace{1+\frac{x}{n}}^n
\end{equation*}
\begin{equation*}
 	e^z = \sum_{n = 0}^{\infty} \frac{z^n}{n!} \\
\end{equation*}
\begin{equation*}
	 \sin (z) = \sum_{n = 0}^{\infty} (-1)^n \frac{z^{2n +1}}{(2n +1)!} = \frac{e^{iz} - e^{-iz}}{2i} \\
\end{equation*}
\begin{equation*}
	 \cos (z) = \sum_{n = 0}^{\infty} (-1)^n \frac{z^{2n}}{(2n)!} = \frac{e^{iz} + e^{-iz}}{2}\\
 \end{equation*}
% subsection potenzreihen (end)





\subsection{Ableitung und Integral}
$f$ diffbar, falls $f$ stetig und $\underset{h\rightarrow 0}{\lim}\frac{f(x_0+h)-f(x_0)}{h}=f'(x_0)$ exist.
\subsubsection{Ableitungsregeln:}
Linearität: $(\lambda f + \mu g)' (x) = \lambda f'(x) + \mu g'(x)$ \quad $\forall \lambda, \mu \in \mathbb R$ \\
Produktregel: $(f \cdot g)' = f' g + f g'$\\
Quotientenregel $\enbrace{\frac{f}{g}}' = \frac{f'g - fg'}{g^2}$\\
Kettenregel: $\left( f(g(x)) \right)' = f'(g(x)) g'(x)$\\
Potenzreihe: $f: ] \underbrace{-R+a, a+R}_{\subseteq D}	 [ \rightarrow \mathbb R, f(x) = \sum_{n=0}^{\infty} a_n (x -a)^n$ \quad $\Rightarrow$ \quad $f'(x) = \sum_{n=0}^{\infty} n a_{n} (x-a)^{n-1}$\\
\textbf{Tangentengleichung:} $y=f(x_0)+f'(x_0)(x-x_0)$

\subsubsection{Newton-Verfahren:}
$x_{n+1}=x_n-\frac{f(x_n)}{f'(x_n)}$ mit Startwert $x_0$

\subsubsection{Integrationsmethoden:}
\begin{itemize}\itemsep0pt
\item Anstarren + Göttliche Eingebung
\item Partielle Integration: $\int uv'=uv-\int u'v$
\item Substitution: $\int f(\underbrace {g(x)}_{t}) \underbrace {g'(x)\,\mathrm dx}_{\mathrm dt}=\int f(t)\, \mathrm dt$
\item Logarithmische Integration: $\int \frac{g'(x)}{g(x)}dx=\ln\abs{g(x)}$
\item Integration von Potenzreihen: $f(x)=\sum_{k=0}^{\infty}a_k(x-a)^k$ \\
Stammfunktion: $F(x)=\sum_{k=0}^{\infty}\frac{a_k}{k+1}(x-a)^{k+1}$
\item Brechstange: $t=\tan(\frac{x}{2})$ \quad $\mathrm dx = \frac{2}{1+t^2} \mathrm dt$ \\ $\sin(x) \rightarrow \frac{2t}{1+t^2}$ \qquad $\cos(x) \rightarrow \frac{1-t^2}{1+t^2}$
\end{itemize}

\subsubsection{Integrationsregeln}
$\int_a^b f(x) \mathrm dx = F(b) - F(a)$\\
$\int\lambda f(x)+\mu g(x) \, \mathrm dx=\lambda\int f(x) \, \mathrm dx + \mu\int g(x) \, \mathrm dx$

\everymath{\displaystyle}	% Formeln ab hier groß Schreiben
\begin{math}\renewcommand{\arraystretch}{1.8}
\begin{array}{c|c|c}
F(x) & f(x) & f'(x) \\ \hline
\frac{1}{q+1}x^{q+1} & x^q & qx^{q-1} \\
\raisebox{-0.2em}{$\frac{2\sqrt{ax^3}}{3}$} & \sqrt{ax} & \raisebox{0.2em}{$\frac{a}{2\sqrt{ax}}$}\\
x\ln(ax) -x & \ln(ax) & \textstyle \frac{1}{x}\\
e^x & e^x & e^x \\
\frac{a^x}{\ln(a)} & a^x & a^x \ln(a) \\
-\cos(x) & \sin(x) & \cos(x)\\
\sin(x) & \cos(x) & -\sin(x)\\
-\ln |\cos(x)| & \tan(x) & \frac{1}{\cos^2(x)} \\
\ln |\sin(x)| & \cot(x) & \frac{-1}{\sin^2(x)} \\
x\arcsin (x)+\sqrt{1-x^2} & \arcsin(x) & \frac{1}{\sqrt{1-x^2}}\\
x\arccos (x)-\sqrt{1-x^2} & \arccos(x) & -\frac{1}{\sqrt{1-x^2}}\\
x\arctan (x)-\frac{1}{2} \ln \left| 1+ x^2 \right| & \arctan (x) & \frac{1}{1+x^2} \\
x\arccot (x)+\frac{1}{2} \ln \left| 1+ x^2 \right| & \arccot (x) & -\frac{1}{1+x^2} \\
x \sinh ^{-1} (x) - \sqrt{x^2+1} & \sinh ^{-1} (x) & \frac{1}{\sqrt{x^2+1}}\\
x \cosh ^{-1} (x) - \sqrt{x^2-1}  & \cosh ^{-1} (x) & \dfrac{1}{\sqrt{x^2-1}}\\
\frac{1}{2}\ln(1-x^2) + x \tanh ^{-1} (x) & \tanh ^{-1} (x) & \frac{1}{1-x^2}\\
\sinh(x) & \cosh(x) & \sinh (x) \\
\cosh(x) & \sinh(x) & \cosh (x)\\
\end{array}
\end{math}
\everymath{\textstyle}


\subsubsection{Rotationskörper}
Volumen: $V = \pi \int_a^b f(x)^2 \mathrm dx$\\
Oberfläche: $O = 2 \pi \int_a^b f(x) \sqrt{1 + f'(x)^2} \mathrm dx$

\subsubsection{Uneigentliche Integrale}
$\int\limits_{\text{ok}}^{\text{böse}} f(x) \mathrm dx = \lim\limits_{b\rightarrow \text{böse}}\ \int\limits_{\text{ok}}^b f(x) \mathrm dx$\\ \\ \\
Majoranten-Kriterium: $|f(x)|\le g(x) = \frac1{x^\alpha}$\\ \\
$\int\limits_{1}^{\infty} \frac{1}{x^\alpha} \mathrm dx \begin{cases} \frac{1}{\alpha -1}, \quad \alpha > 1 \\ \infty, \qquad \alpha \le 1 \end{cases}$ \qquad
$\int\limits_{0}^{1} \frac{1}{x^\alpha} \mathrm dx \begin{cases} \frac{1}{\alpha -1}, \quad \alpha < 1 \\ \infty, \qquad \alpha \ge 1 \end{cases}$\\ \\
\textbf{Cauchy-Hauptwert}
\begin{eqnarray*}
& \text{CHW }\int\limits_{-\infty}^{\infty} f(x) \mathrm dx = \lim\limits_{b\rightarrow\infty} \int\limits_{-b}^b f(x) \mathrm dx \\
& \text{CHW }\int\limits_{a}^{b} f(x) \mathrm dx=\lim\limits_{\varepsilon \rightarrow 0^+}\enbrace{\int\limits_{a}^{c-\varepsilon} f(x) \mathrm dx+\int\limits_{c+\varepsilon}^{b}f(x)\mathrm dx}
\end{eqnarray*}

\subsubsection{Laplace-Transformation von $f: [0,\infty[ \ra \mathbb R,\ s \mapsto f(s)$}
$\mathcal L \; f(s) = F(s) = \int\limits_{0}^{\infty} e^{-st} f(t)\ \mathrm dt = \lim\limits_{b \ra \infty} \int\limits_{0}^{b} e^{-st} f(t)\ \mathrm dt$

\subsubsection{Integration rationale Funktionen}
Gegeben: $\int \frac{A(x)}{Q(x)} \mathrm dx \qquad A(x),Q(x)\in \mathbb R[x]$
\begin{enumerate}\itemsep0pt
\item Falls, $\deg A(x) \ge \deg Q(x) \Ra$ Polynomdivision: \\ $\frac{A(x)}{Q(x)} = P(x) + \frac{B(x)}{Q(x)}$ mit $\deg B(x) < \deg Q(x)$
\item Zerlege $Q(x)$ in unzerlegbare Polynome
\item Partialbruchzerlegung $\frac{B(x)}{Q(x)} = \frac{\ldots}{(x - a_n)} + \ldots + \frac{\ldots}{\ldots}$
\item Integriere die Summanden mit folgenden Funktionen
\end{enumerate}

$\text{mit} ~ \lambda=x^2+px+q, ~~ \beta=4q-p^2 ~~ \text{und} ~p^2<4q$!
$\int\frac{1}{(x-a)^m}\mathrm dx \begin{cases} \ln\left|x-a\right|, & m=1\\ \\ \frac{-1}{(m-1)(x-a)^{m-1}} &m\geq2 \end{cases}$\\ \\ \\
$\int\frac{1}{(\lambda)^m} \mathrm dx \begin{cases} \frac{2}{\sqrt{\beta}} \arctan\frac{2x+p}{\sqrt{\beta}}, &m=1\\ \\ \frac{2x+p}{(m-1)(\beta)(\lambda)^{m-1}}+\frac{2(2m-3)}{(m-1)(\beta)} \int\frac{\mathrm dx}{(\lambda)^{m-1}}, &m\geq2 \end{cases}$\\ \\ \\
$\int\frac{Bx+C}{(\lambda)^m} \mathrm dx \begin{cases} \frac{B}{2} \ln(\lambda) + (C-\frac{Bp}{2}) \int\frac{\mathrm dx}{\lambda}, &m=1\\ \\ \frac{-B}{2(m-1)(\lambda)^{m-1}} + (C-\frac{Bp}{2}) \int\frac{\mathrm dx}{(\lambda)^{m-1}}, &m\geq2 \end{cases}$\\ \\ \\
Häufige Integrale nach Partialbruchzerlegung
\begin{align*}
&\int\frac{1}{x}\mathrm dx = \ln\abs{x}&&\int\frac{1}{x^2}\mathrm dx = -\frac{1}{x} \\
&\int\frac{1}{a+x}\mathrm dx = \ln\abs{a+x} && \int\frac{1}{(a+x)^2}\mathrm dx = -\frac{1}{a+x} \\
&\int\frac{1}{a-x}\mathrm dx = -\ln\abs{a-x} && \int\frac{1}{(a-x)^2}\mathrm dx = \frac{1}{a-x}
\end{align*}
\subsubsection{Paratialbruchzerlegung}
\begin{equation*}
\frac{B(x)}{Q(x)} = \frac{\ldots}{(x - x_0)} + \ldots + \frac{\ldots}{\ldots}
\end{equation*}
\paragraph{Ansatz}
\begin{itemize}\itemsep0pt
\item $n$-fache reelle Nullstelle $x_0$: $\frac{A}{x-x_0}+\frac{B}{(x-x_0)^2} +\dots$
\item $n$-fache komplexe Nullstelle: $\frac{Ax+B}{x^2+px+q}+\frac{Ax+B}{(x^2+px+q)^2}$
\end{itemize}
\paragraph{Berechnung von $A,B,C,\dots$}
\begin{itemize}\itemsep0pt
\item Nullstellen in $x$ einsetzen (Terme fallen weg)
\item Ausmultiplizieren und Koeffizientenvergleich
\end{itemize}


\subsection{Taylor-Entwicklung}
%===========================================================================================================================================================
Man approximiert eine $m$-mal diffbare Funktion $f:I=[a,b] \rightarrow \mathbb R$ \\ in $x_0 \in I$ mit dem $m$-ten Taylorpolynom:\\
\begin{equation*}
T_{m}(x_0;x)= \sum_{i=0}^{m} \frac{f^{(i)}(x_0)}{i!}(x-x_0)^i
\end{equation*}
%$T_{m,f,x_0}(x)=f(x_0)+\frac{f'(x_0)}{1!}(x-x_0)+\frac{f''(x_0)}{2!}(x-x_0)^2+\dotsc +\frac{f^{(m)}(x_0)}{m!}(x-x_0)^m$\\
Taylor-Entw. von Polynomen/Potenzreihen sind die Funktionen selbst.\\
Für $m \ra \infty$: Taylorreihe. \\
Konvergenzradius: $R = \underset{n\rightarrow \infty}{\lim} \abs{\frac{a_n}{a_{n+1}}}=\lim\limits_{n\rightarrow \infty}\frac{1}{\sqrt[n]{\abs{a_n}}}$

	\subsubsection{Das Restglied - die Taylorformel}
	Für $(m+1)$-mal stetig diffbare Funktionen gilt $\forall x \in I:$\\
	$R_{m+1}(x) := f(x)- T_{m,f,x_0}(x) =$  \\
	$= \frac{1}{m!} \int_{x_0}^{x}(x-t)^m f^{(m+1)}(t)\mathrm dt$ \ \  (Integraldarst.)\\
	$= \frac{f^{(m+1)}(\xi)}{(m+1)!}(x-x_0)^{m+1}$ \quad
	$\xi \in [x, x_0]$ (Lagrange) \\
	\textbf{Fehlerabschätzung:} Wähle $\xi$ und $x$ so, dass $R_{m+1}(x)$ maximal wird.


	\subsection{Elementarfunktionen}
	\begin{itemize}\itemsep-1pt
		\item Exponentialfunktion\\
		$e^x = \sum\limits_{k = 0}^m\frac{x^k}{k!}$
		\item Trigonometrische Funktionen\\
		$\sin{x} = \sum\limits_{k = 0}^m(-1)^k\frac{x^{2k + 1}}{(2k + 1)!}$\\
		$\cos{x} = \sum\limits_{k = 0}^m(-1)^k\frac{x^{2k}}{(2k)!}$
		\item Logarithmusfunktion\\
		$\ln{(1 + x)} = \sum\limits_{k = 1}^m\frac{(-1)^{k + 1}}{k}x^k$
	\end{itemize}




\vspace{10em}



\end{multicols*}
% Ende der Spalten


% Dokumentende
% ======================================================================
\end{document}