
% Authors: Emanuel Regnath, Martin Zellner
% Contact: info@latex4ei.de
% Encode: UTF-8, tabwidth = 4, newline = LF
% % % % % % % % % % % % % % % % % % % % % % % % % % % % % % % % % % % % % % % %


% ======================================================================
% Document Settings
% ======================================================================

% possible options: color/nocolor, english/german, threecolumn
% defaults: color, english
\documentclass[german]{latex4ei/latex4ei_sheet}

% set document information
\title{Mathematik}
%\author{Daniel Gloukhman}					% optional, delete if unchanged
%\myemail{danielgloukhman@hotmail.de}			% optional, delete if unchanged
%\mywebsite{www.latex4ei.de}			% optional, delete if unchanged


% ======================================================================
% Begin
% ======================================================================
\begin{document}


% Title
% ----------------------------------------------------------------------
\maketitle   % requires ./img/Logo.pdf


% Section
% ----------------------------------------------------------------------
\section{Allgemeine Formeln/Ungleichungen}
\subsection{Allgemeines} % (fold)
\label{sub:allgemeines}
\begin{sectionbox}

Dreiecksungleichung \qquad \qquad \qquad
\begin{math}\begin{array}{l}
	\abs{x + y} \le \abs{x} + \abs{y} \\
	\abs{\abs{x}- \abs{y}} \le \abs{x-y}
\end{array}\end{math} \\
Cauchy-Schwarz-Ungleichung: \qquad
\begin{math}\begin{array}{l}
\left| \sprod{x}{y} \right| \le \| x\| \cdot \| y\|
\end{array}\end{math} \\
\\
Arithmetische Summenformel \qquad
\begin{math}\begin{array}{l}
	\sum\limits_{k = 1}^{n}k = \frac{n (n+1)}{2}
\end{array}\end{math}  \\
\\
Geometrische Summenformel \qquad
\begin{math}\begin{array}{l}
	\sum\limits_{k = 0}^{n}q^k = \frac{1 - q^{n+1}}{1-q}
\end{array}\end{math}\\
\\
Bernoulli-Ungleichung \qquad \qquad \quad
\begin{math}\begin{array}{l}
	(1+a)^n \ge 1 + na
\end{array}\end{math}\\
\\
Binomialkoeffizient \qquad \qquad \qquad
\begin{math}\begin{array}{l}
	\binom{n}{k} = \frac{n!}{k!(n-k)!}  \\
	\binom{n}{0} = \binom{n}{n} = 1
\end{array}\end{math}\\
\\
Binomische Formel \qquad \qquad \qquad
\begin{math}\begin{array}{l}
	(a+b)^n = \sum\limits_{k = 0}^{n} \binom{n}{k} a^{n-k} b^{k}
\end{array}\end{math}   \\
\\
Äquivalenz von Masse und Energie
\begin{math}\begin{array}{l}
	E = mc^2
\end{array}\end{math}\\
\\
Wichtige Zahlen: $\sqrt{2} = 1,41421$\quad $\pi=$ ist genau 3 \quad $e = 2,71828$ \quad $\pi =  3,14159$

\paragraph{Fakultäten} % (fold)
\label{par:fakultaeten}
$n! = 1 \cdot 2 \cdot 3 \cdot \ldots \cdot n$ \qquad  $0! = 1! = 1$

% paragraph fakultäten (end)
% subsubsection subsection_name (end)
% subsection allgemeines (end)


\end{sectionbox}




\section{Komplexe Zahlen}
\begin{sectionbox}
    \subsection{Komplexe Zahlen}
    % ----------------------------------------------------------------------
    Eine komplexe Zahl $z=a+b\mathbf{i},\ z\in \mathbb C, \quad a,b \in \mathbb R$ besteht aus einem Realteil $\Re(z)=a$ und einem Imaginärteil $\Im(z)=b$, wobei $\mathbf{i}=\sqrt{-1}$ die immaginären Einheit ist.
    Es gilt: \quad $i^2 = -1$ \quad $i^4 = 1$
    \subsubsection{Kartesische Koordinaten}
    Rechenregeln:\\
    $z_1+z_2=(a_1+a_2)+(b_1+b_2)\mathbf{i}$\\
    $z_1\cdot z_2=(a_1\cdot a_2-b_1\cdot b_2)+(a_1\cdot b_2+a_2\cdot b_1)\mathbf{i}$\\
    \\
    Konjugiertes Element von $z=a+b\mathbf{i}$:\\
    $\overline{z}=a-b\mathbf{i}$\qquad \qquad \qquad \qquad \qquad \qquad \qquad \qquad $e^{\overline{ix}} = e^{-ix}$  \\
    $z\overline{z}=|z|^2=a^2+b^2$\\
    \\
    Inverses Element:\\
    $z^{-1}=\frac{1}{z}\frac{\overline z}{\overline z z}=\frac{\overline z}{a^2+b^2}=\frac{a}{a^2+b^2} - \frac{b}{a^2+b^2}\mathbf{i}$
    
    
    \subsubsection{Polarkoordinaten}
    $z=a+b\mathbf{i}\ne0$\ in Polarkoordinaten:\\
    $z=r (\cos(\varphi)+\mathbf{i}\sin(\varphi))=r\cdot e^{\mathbf{i} \varphi}$\\
    $r=|z|=\sqrt{a^2+b^2}\quad\varphi=\arg(z)=\begin{cases}+\arccos \left( \frac{a}{r}\right),  & b\ge0   \\  -\arccos \left( \frac{a}{r}\right), & b<0\end{cases}$
    
    \begin{description}\itemsep0pt
    \item[Multiplikation:] $z_1\cdot z_2=r_1 \cdot r_2 ( \cos ( \varphi_1 + \varphi_2) + \mathbf{i} \sin (\varphi_1 + \varphi_2))$
    \item[Division:] $\frac{z_1}{z_2}=\frac{r_1}{r_2} ( \cos ( \varphi_1 - \varphi_2) + \mathbf{i} \sin (\varphi_1 - \varphi_2))$
    \item[n-te Potenz:] $z^n=r^n\cdot e^{n\varphi \mathbf{i}}= r^n (\cos (n \varphi) + \mathbf{i} \sin (n \varphi))$
    \item[n-te Wurzel:] $\sqrt[n]{z}= z_k = \sqrt[n]{r} \left(\cos \left(\frac{\varphi + 2k\pi}{n}\right) + \mathbf{i} \sin \left(\frac{\varphi + 2k\pi}{n}\right)\right) \\ k =0,1, \ldots, n-1$
    \item[Logarithmus:] $\ln(z)=\ln(r) + \mathbf{i}(\varphi + 2k\pi)$ \quad (Nicht eindeutig!)
    \end{description}
    Anmerkung: Addition in kartesische Koordinaten umrechnen(leichter)!
    
    
\end{sectionbox}





\begin{sectionbox}
    \subsection{Funktionen}
    Eine Funktion $f$ ist eine Abbildung, die jedem Element $x$ einer Definitionsmenge $D$ genau ein Element $y$ einer Wertemenge $W$ zuordnet.\\
    $f:D\rightarrow W,\ x \mapsto f(x):=y$\\
    \\
    \textbf{Injektiv}: $f(x_1)=f(x_2) \Rightarrow x_1=x_2$\\
    \textbf{Surjektiv}: $\forall y\in W \exists x\in D:f(x)=y$\\ \quad (Alle Werte aus $W$ werden angenommen.)\\
    \textbf{Bijektiv}(Eineindeutig): $f$ ist injektiv und surjektiv $\Rightarrow$ $f$ umkehrbar. \\
    \textbf{Ableitung der Umkehrfunktion} \\
    $f$ stetig, streng monoton, an $x_0$ diff'bar und $y_0=f(x_0) \\
    \Rightarrow \enbrace{f^{-1}}(y_0)= \frac{1}{f'(x_0)} =\frac{1}{f'(f^{-1}\enbrace{y_0})}$
    
    \subsubsection{Symmetrie einer Funktion $f$}
    \textbf{Achsensymmetrie} (gerade Funktion): $f(-x)=f(x)$\\
    \textbf{Punktsymmetrie} (ungerade Funktion): $f(-x)=-f(x)$\\
    \\
    Regeln für gerade Funktion $g$ und ungerade Funktion $u$:\\
    $g_1 \pm g_2 = g_3$ \qquad $u_1 \pm u_2 = u_3$\\
    $g_1 \cdot g_2=g_3$ \qquad $u_1 \cdot u_2 = g_3$ \qquad $u_1 \cdot g_1=u_3$
    
    \subsubsection{Kurvendiskussion von $f: I = [a, b] \rightarrow \mathbb{R}$}
    \textbf{Kandidaten für Extrama (lokal, global)}
    \begin{enumerate}\itemsep0pt
    \item Randpunkte von $I$
    \item Punkte in denen $f$ nicht diffbar ist
    \item Stationäre Punkte ($f'(x)=0$) aus $(a,b)$
    \end{enumerate}
    \textbf{Lokales Maximum} \\
    wenn $x_0$ stationärer Punkt ($f'(x_0) = 0$) und
    \begin{itemize}\itemsep0pt
    \item
    $f''(x_0)<0$ oder
    \item
    $f'(x) > 0, x \in (x_0-\varepsilon, x_0) \\
     f'(x) < 0, x \in (x_0, x_0+\varepsilon)$
    \end{itemize}
    \textbf{Lokales Minimum} \\
    wenn $x_0$ stationärer Punkt ($f'(x_0) = 0$) und
    \begin{itemize}\itemsep0pt
    \item
    $f''(x_0)>0$ oder
    \item
    $f'(x) < 0, x \in (x_0-\varepsilon, x_0) \\
     f'(x) > 0, x \in (x_0, x_0+\varepsilon)$
    \end{itemize}
    %$f'(x_0) = 0 \quad \begin{cases}f''(x_0)<0 \ \rightarrow \ \text{Maximum (lokal)} \\ f''(x_0)>0 \ \rightarrow \ \text{Minimum (lokal)} \end{cases} $\\
    \textbf{Monotonie} \\
    $f'(x) \underset{(>)}{^{\ge}} 0 \rightarrow$ \ $f$ (streng) Monoton steigend, $x\in(a,b)$\\
    $f'(x) \underset{(<)}{^{\le}} 0 \rightarrow$ \ $f$ (streng) Monoton fallend, $x\in(a,b)$\\
    \textbf{Konvex/Konkav} \\
    $f''(x) \underset{(>)}{^{\ge}} 0 \ \rightarrow$ \ $f$ (strikt) konvex, $x\in(a,b)$\\
    $f''(x) \underset{(<)}{^{\le}} 0 \ \rightarrow$ \ $f$ (strikt) konkav, $x\in(a,b)$\\
    $f''(x_0)=0 \text{ und } f'''(x_0) \ne 0 \rightarrow x_0$ Wendepunkt \\
    $f''(x_0)=0 \text{ und Vorzeichenwechseln an } x_0 \rightarrow x_0$ Wendepunkt \\
    \subsubsection{Asymptoten von $f$}
    Horizontal: $c=\lim\limits_{x\ra \pm \infty} f(x)$\\
    Vertikal: $\exists \text{ Nullstelle } a \text{ des Nenners }: \lim\limits_{x \rightarrow a^{\pm}} f(x) = \pm \infty$\\
    Polynomasymptote $P(x)$: $f(x):=\frac{A(x)}{Q(x)}=P(x)+ \underset{\ra 0}{\frac{B(x)}{Q(x)}}$ \\
    \subsubsection{Wichtige Sätze für \underline{stetige} Fkt. $f: [a,b] \rightarrow \mathbb R, f \mapsto f(x)$ }
    \textbf{Zwischenwertsatz:} $\forall y \in [f(a),f(b)]\ \exists x\in [a,b]:f(x)=y$\\
    \textbf{Satz von Rolle:} Falls $f(a)=f(b)$, dann $\exists x_0: f' (x_0) = 0$\\
    \textbf{Mittelwertsatz:} Falls $f$ diffbar, dann $\exists x_0:f'(x_0)=\frac{f(b)-f(a)}{b-a}$\\
    \textbf{Regel von L'Hospital}:\\ $\lim\limits_{x \rightarrow a} \frac{f(x)}{g(x)} = \left[ \frac{0}{0} \right] / \left[ \frac{\infty}{\infty} \right] \rightarrow \lim\limits_{x \rightarrow a} \frac{f(x)}{g(x)} = \lim\limits_{x \rightarrow a} \frac{f'(x)}{g'(x)}$
    
    % ----------------------------------------------------------------------
    \subsubsection{Polynome $P(x)\in\mathbb R[x]_n$}
    $P(x)=\sum_{i=0}^n a_ix^i=a_n x^n+a_{n-1} x^{n-1}+\dotsc+a_1x+a_0$ \\
    Lösungen für $ax^2+bx+c=0$ \\
    \begin{tabular}{l|l}
    \textbf{Mitternachtsformel}:  &  Satz von Vieta:\\
    $x_{1/2}=\frac{-b\pm\sqrt{b^2-4ac}}{2a}$  \quad & \quad   $x_1 + x_2 = - \frac{b}{a} \qquad x_1 x_2 = \frac{c}{a}$
    \end{tabular}
    \breakline
    \subsubsection{Trigonometrische Funktionen}
    $f(t)=A\cdot \cos(\omega t + \varphi_0)=A\cdot \sin(\omega t + \frac{\pi}{2}+ \varphi_0)$
    \begin{eqnarray*}
        \sin (-x) = -\sin (x)  \quad & \quad \cos (-x) = \cos (x) \\
        \sin^2 x + \cos^2 x = 1  \quad & \quad \tan x = \frac{\sin x}{\cos x} \\
        e^{ix}=\cos(x)+i\sin(x) \quad & \quad e^{-ix}=\cos(x)-i\sin(x) \\
        \sin(x)=\frac{1}{2i}\enbrace{e^{ix}-e^{-ix}} \quad & \quad \cos(x)=\frac{1}{2}\enbrace{e^{ix}+e^{-ix}} \\
        \sinh(x)=\frac{1}{2}(-e^{-x}+e^x) & \cosh(x)=\frac{1}{2}(e^{-x}+e^x)
    \end{eqnarray*}
    \paragraph{Additionstheoreme} % (fold)
    \label{par:additionstheoreme}
     \begin{eqnarray*}
         & \cos (x + y) = \cos x \cos y - \sin x \sin y \\
        & \cos \enbrace{x - \frac{\pi}{2}} = \sin x \qquad \quad \sin \enbrace{x + \frac{\pi}{2}} = \cos x \\
        & \sin \enbrace{x + y} = \sin x \cos y + \cos x \sin y \\
    & 	\sin 2x = 2 \sin x \cos x        \\
        & \cos 2x = \cos^2 x - \sin^2 x = 2\cos^2 x - 1\\
     \end{eqnarray*}
    % paragraph additionstheoreme (end)
    \hspace{-20pt}
    \scalebox{0.77}
    {
    $\begin{array}{c|c|c|c|c|c|c|c|c|c|c|c}
    x & 0& 30 & 45& 60 & 90 & 120 & 135& 150& 180 & 270 & 360 \\ \hline
    x & 0 & \pi / 6 & \pi / 4 & \pi / 3 & \pi / 2 & \frac{2}{3}\pi& \frac{3}{4}\pi& \frac{5}{6}\pi& \pi  & \frac{3}{2}\pi & 2 \pi \\ \hline
    \sin & 0 & \frac{1}{2} & \frac{1}{\sqrt{2}} & \frac{\sqrt 3}{2} & 1 & \frac{\sqrt 3}{2} & \frac{1}{\sqrt{2}} & \frac{1}{2} & 0 & -1 & 0 \\
    \cos & 1 & \frac{\sqrt 3}{2} & \frac{1}{\sqrt 2} & \frac{1}{2} & 0 & -\frac{1}{2} & -\frac{1}{\sqrt 2}  & -\frac{\sqrt 3}{2}   & -1 & 0 & 1 \\
    \tan & 0 & \frac{\sqrt{3}}{3}&1 &\sqrt{3} & \lightning & -\sqrt{3}& -1& -\frac{1}{\sqrt{3}} & 0 & \lightning & 0\\
    \end{array}$
    }
    
    \subsubsection{Potenzen/Logarithmus}
    \begin{equation*}
    \ln(u^r)=r\ln u
    \end{equation*}
    
    
\end{sectionbox}

\section{Zahlenfolgen}


\begin{sectionbox}
	\subsection{Konvergenz}
	Eine Folge $(a_n)$ konvergiert gegen den Grenzwert a



	$\forall\mathcal{E}>0 \text{ }\exists N \in \R \text{ }\forall n \in \N:\text{ } n>N \Rightarrow |a_n - a|<\mathcal{E}$





\end{sectionbox}

\begin{sectionbox}
	\subsection{Monotoniesatz}

	Jede beschränkte und monotone Folge ist konvergent

\end{sectionbox}



\begin{sectionbox}
	\subsection{Einschließungskriterium}

Seien $(a_n), (b_n)$  reelle Folgen mit $a = \lim\limits_{n \to \infty}  a_n =  \lim\limits_{n \to \infty}b_n $ , und eine dritte reelle Folge $(c_n)$ erfülle  $(a_n) \le (c_n) \le (b_n)$ für fast alle $n$. Dann konvergiert auch \(c_n\)  gegen $a$.


\end{sectionbox}

\begin{sectionbox}
	\subsection{Bestimmte Divergenz}

	\(a_n\) ist bestimmt divergent gegen $+\infty$ falls gilt: \\
	$\forall M >0 \text{ }\exists N \in \R \text{ }\forall n \in \N: n>N \Rightarrow a_n>M$ \\
Für $a_n \in \C $ muss $|a_n| \rightarrow +\infty $ gelten
\end{sectionbox}


\begin{sectionbox}
	\subsection{Cauchy-Folge}
	Jede konvergente Folge ist eine Cauchy-Folge \\
	$	\forall\mathcal{E}>0 \text{ }\exists N \in \R \text{ }\forall m,n \in \N:\text{ } m,n>N \Rightarrow |a_m - a_n|<\mathcal{E}$

\end{sectionbox}

\begin{sectionbox}
	\subsection{Bekannte Grenzwerte}
	Für jedes $\beta > 0$ gilt:
	\begin{equation*}
		\lim \limits_{x \to \infty}(x^\beta e^{-x})= 0, \lim \limits_{x \to \infty} \frac{ln(x)}{x^\beta} = 0, \lim \limits_{x \to 0^+}(x^\beta ln(x))= 0
	\end{equation*}

\end{sectionbox}





\section{Reihen}

\begin{sectionbox}
	\subsection{Geometrische Reihe}
	Die Reihe $\sum \limits_{k=0}^{\infty}\ q^k$ konvergiert für $|q|<1$ und divergiert andernfalls. Es gilt:\\
	$\sum \limits_{k=0}^{\infty}\ q^k = \frac{1}{1-q}$

\end{sectionbox}

\begin{sectionbox}
	\subsection{Quotientenkriterium}
	Sei $\sum \limits_{k=0}^{\infty}\ a_k$ und existiere
	$q := \lim\limits_{k \to \infty} |\frac{a_{k+1}}{a_k}| $ dann gilt: \\
	- für $q<1$ ist die Reihe absolut konvergent\\
	- für $q>1$ ist die Reihe divergent

\end{sectionbox}

\begin{sectionbox}
	\subsection{Wurzelkriterium}

		Sei $\sum \limits_{k=0}^{\infty}\ a_k$ und existiere
	$q := \limsup \limits_{k \to \infty} \sqrt[k]{|a_k|}\ $ dann gilt: \\
	- für $q<1$ ist die Reihe absolut konvergent\\
	- für $q>1$ ist die Reihe divergent

\end{sectionbox}

\begin{sectionbox}
	\subsection{Leibnizkriterium}
	Sei $(a_n)$ eine monoton fallende Nullfolge.\\Die  Reihe $s = \sum \limits_{k=1}^{\infty}\ (-1)^{k-1} a_k$ konvergiert.\\
	Es gilt: $|s - \sum \limits_{k=1}^{n}\ (-1)^{k-1} a_k| \le a_{n+1}$




\end{sectionbox}


\begin{sectionbox}
	\subsection{Majoranten/Minorantenkriterium}
	Sei $\sum a_k$  und $\sum b_k$ zwei Reihen.\\
	%$|a_k| \le b_k$
	\\1. Gilt $0 \le |a_k| \le b_k$ für fast alle k, und ist die Majorante $\sum  b_k$ konvergent, so konvergiert $\sum  a_k$ absolut. \\
    \\2. Gilt $0 \le a_k \le b_k$ für fast alle k, und ist die Minorante $\sum  a_k$ divergent, so divergiert $\sum  b_k$

	%\\ Wenn $\sum \limits_{k=1}^{\infty}\ b_k$ konvergiert, konvergiert  $\sum \limits_{k=0}^{\infty}\ a_k$ absolut.\\
	%$|\sum \limits_{k=0}^{\infty}\ a_k| \le \sum \limits_{k=0}^{\infty}\ b_k$

\end{sectionbox}
\begin{sectionbox}
	\subsection{Integralvergleichskriterium}

	Sei $f:[1, \infty) \rightarrow [0,\infty)$ monoton fallend, dann konvergiert die Reihe $\sum \limits_{k=1}^{\infty} f(k)$ genau dann, wenn das uneigentliche Integral $\int_{1}^{\infty}f(x)dx$ konvergiert.
\end{sectionbox}


\section{Potenzreihen}

\begin{sectionbox}
	\subsection{Definition}
	Sei $(a_k)_{k \in \N},z \in \C$,  $\rho \in \R \cup \{\infty\} $ der Konvergenzradius und $z_0$ ein Entwicklungspunkt.\\
	Die Potenzreihe $P(z)=\sum \limits_{k=0}^{\infty} a_k(z-z_0)^k$ konvergiert absolut für $|z| \le \rho$.
\end{sectionbox}

\begin{sectionbox}
	\subsection{Konvergenradius}
	Man betrachte die Koeffizienten Folge $(a_k)$.\\
	\begin{itemize}
	\item $\rho := \frac{1}{\limsup\limits_{k \to \infty}\sqrt[k]{|a_k|}}$ mit ,,$\frac{1}{0}=+\infty$" \text{ und } ,,$\frac{1}{\infty}=0$"
	\item  $\rho := \frac{1}{\lim\limits_{k \to \infty}|\frac{a_{k+1}}{a_k}|} $ mit ,,$\frac{1}{0}=\infty$"
	\end{itemize}

\end{sectionbox}

\begin{sectionbox}
	\subsection{Cauchy-Produkt}
	Seien $(a_k),(b_l)$ Folgen, das Cauchy-Produkt $c=a*b$ ist die neue Folge $c_m = \sum\limits_{k=0}^{m}a_k b_{m-k}$\\ \\
	Konvergieren die Reihen von $\alpha,\beta$ absolut, konvergiert auch  $\sum_{m}^{\infty}(\alpha * \beta)$ absolut und es gilt: \\
	$\sum\limits_{m=0}^{\infty}(\alpha * \beta)_m =(\sum\limits_{k=0}^{\infty}\alpha_k)(\sum\limits_{l=0}^{\infty}\beta_l)$

\end{sectionbox}

\begin{sectionbox}
	\subsection{Exponentialfunktion}
	$exp:\C\rightarrow\C$\\
	$exp(z)=e^z=\sum\limits_{k=0}^{\infty}\frac{z^k}{k!} \text{ für alle } z \in \C$\\
	$e^{z+w}=e^ze^w$\\
	Umkehrfunktion: $ln(xy)=ln(x)+ln(y)$

\end{sectionbox}


\begin{sectionbox}
	\subsection{Sinus und Cosinus}
	\begin{itemize}

	\item $sin(z):=\frac{e^{iz}-e^{-iz}}{2i}=\sum\limits_{k=0}^{\infty}(-1)^k \frac{z^{2k+1}}{(2k+1)!}$
\item	$cos(z):=\frac{e^{iz}+e^{-iz}}{2}=\sum\limits_{k=0}^{\infty}(-1)^k \frac{z^{2k}}{(2k)!}$

	\item \textbf{Eulersche Formel: }$e^{iz}=cos(z)+i \text{ }sin(z)$
	\item \textbf{trigonometrischer Pythagoras: }$sin^2(z)+cos^2(z)=1$
	\item \textbf{Paritäten: } $sin(-z)=-sin(z) \text{ und } cos(-z)=cos(z)$
	\item \textbf{Additionstheoreme:} \begin{align*}
		sin(z+w) &= sin(z)cos(w)+cos(z)sin(w)\\
		cos(z+w) &= cos(z)cos(w)-sin(z)sin(w)
		\end{align*}
\item \textbf{Ableitungen: } $sin'(x)=cos(x), cos'(x)=-sin(x)$


	\end{itemize}





\end{sectionbox}

\begin{sectionbox}
	\subsection{Sinus und Cosinus Hyperbolicus}
	\begin{itemize}
		\item $sinh(z):=-i \text{  } sin(iz)=\frac{e^z-e^{-z}}{2}$
		\item $cosh(z):=cos(iz)=\frac{e^z+e^{-z}}{2}$
	    \item \textbf{hyperbolischer Pythagoras: }$cosh^2(z)-sinh^2(z)=1$
	    \item \textbf{Paritäten: } $sinh(-z)=-sinh(z) \text{ und } cosh(-z)=cosh(z)$
     	\item \textbf{Additionstheoreme:} \begin{align*}
		sinh(z+w) &= sinh(z)cosh(w)+cosh(z)sinh(w)\\
		cosh(z+w) &= cosh(z)cosh(w)+sinh(z)sinh(w)
		\end{align*}
		\item \textbf{Ableitungen: } $sinh'(x)= cosh(x), cosh'(x) = sinh(x)$

	\end{itemize}

\end{sectionbox}





\section{Stetigkeit}

\begin{sectionbox}
	\subsection{Definition}
	$f:\X \rightarrow \Y \text{ heißt stetig am Punkt } a\in \X \text{ und stetig wenn }f \text{  } \forall a \in \X \text{ stetig ist.}$\\ \\
	$\forall \mathcal{E}>0 \text{  } \exists \delta >0 \text{  } \forall x \in \X : |x-a|<\delta \Rightarrow |f(x)-f(a)|< \mathcal{E}$


\end{sectionbox}

\begin{sectionbox}
	\subsection{Folgenstetigkeit}
	Sei $(x_n)$ eine gegen $x$ konvergente Folge.$f$ ist genau dann stetig, wenn\\ \\
	$\lim \limits_{n \to \infty}f(x_n) = f(\lim \limits_{n \to \infty} x_n) = f(x)$

\end{sectionbox}


\begin{sectionbox}
	\subsection{Lipschitz-Stetigkeit}
	$f:\X \rightarrow \Y$ ist Lipschitz-stetig wenn ein $L \in \R_{>0}$ existiert sodass $\forall x_1,x_2 \in \X$\\
	$|f(x_1)-f(x_2)| \le L |x_1-x_2|$\\
	und heißt lokal Lipschitz-stetig, wenn es zu jeder kompakten Menge $K \subset X$ eine lokale Lipschitz-Konstante $L$ gibt, die obiges erfüllt.

\end{sectionbox}

\begin{sectionbox}
	\subsection{Grenzwert von Funktionen}
	Für $f:\X \backslash \{a\} \rightarrow \Y, \text{ ist } y \in \Y $ der Grenzwert falls für alle Folgen $(x_n)$ die gegen $a$ konvergieren gilt $\lim \limits_{x \to a}f(x)=y$
	%Sei $(x_n)$ jede beliebige gegen $a$ konvergent Folge

\end{sectionbox}

\begin{sectionbox}
	\subsection{Satz vom Minimum und Maximum}
	Seien $f : \X \rightarrow \R$ eine stetige reelle Funktion und $A \subset X$ kompakt. Dann existieren $x_{min},x_{max} \in A$ mit $f(x_{max}) = \text{max} (f(A)) \text{ und } f(x_{min}) = \text{min} (f(A))$.

\end{sectionbox}

\begin{sectionbox}
	\subsection{Zwischenwertsatz}
	Es sei $f : [\alpha,\beta] \rightarrow \R$ eine stetige Funktion mit $f(\alpha) < f(\beta)$. Weiter sei $y \in [f(\alpha),f(\beta)]$ . Dann existiert ein $x_y \in [\alpha,\beta]$ mit $f(x_y) = y$.

\end{sectionbox}


\begin{sectionbox}
	\subsection{Punktweise Konvergenz von Funktionenfolgen}
	Die Funktionenfolge $f_n : \X \rightarrow \Y$ heißt punktweise konvergent gegen eine Funktion $f : \X \rightarrow \Y$ wenn für alle $x \in X$ gilt: $\lim \limits_{n \to \infty} f_n(x)=f(x)$

	\end{sectionbox}


\begin{sectionbox}
	\subsection{Gleichmäßige Konvergenz von Funktionenfolgen}
	Eine Folge $(f_n)$ von Funktionen $f_n : \X \rightarrow \Y$ konvergiert
gleichmäßig  gegen eine Funktion $f : \X \rightarrow \Y$ , falls \\
	$\forall \mathcal{E}>0 \text{  } \exists N \in \R \text{   } \forall n > N \text{   } \forall x \in \X: |f_n(x)-f(x)|<\mathcal{E}$\\
	$\Leftrightarrow \lim \limits_{n \to \infty} ||f_n-f ||_\infty = 0$


\end{sectionbox}
\begin{sectionbox}
	\subsection{Stetigkeit von Grenzwertfunktionen}
		Konvergiert die Folge $(f_n)$ stetiger Funktionen $f_n : \X \rightarrow \Y$ gleichmäßig gegen $f :\X \rightarrow \Y$, so ist $f$ stetig.

\end{sectionbox}




\section{Differenzierbarkeit}

\begin{sectionbox}
	\subsection{Differentialquotient}
	$f: I \rightarrow \C$ ist differenzierbar bei $x_* \in I $ wenn der Grenzwert existiert \\
	$f'(x*)= \lim \limits_{x \to x_*} \frac{f(x)-f(x_*)}{x-x_*}$

\end{sectionbox}

\begin{sectionbox}
	\subsection{Differenzierbar impliziert Stetig}
	Ist $f: I \rightarrow \C$ differenzierbar bei $ x_* \in I$ dann ist f auch stetig bei $x_*$ \\
	($\neg$ Stetig $\Rightarrow \neg$ Differenzierbar)

\end{sectionbox}

\begin{sectionbox}
	\subsection{Ableitungsregeln}
	\begin{itemize}
		\item \textbf{Linearität: } $(\lambda f + \mu g)'(x)=\lambda f'(x) + \mu g'(x)$
		\item \textbf{Produktregel: } $(fg)'(x)=f'(x)g(x)+f(x)g'(x)$
		\item \textbf{Quotientenregel: } $(\frac{f}{g})'(x)= \frac{f'(x)g(x)-f(x)g'(x)}{(g(x))^2}$
		\item \textbf{Kettenregel: } $(f \circ g)'(x)=f'(g(x))g'(x)$
	\end{itemize}

\end{sectionbox}

\begin{sectionbox}
	\subsection{Ableitung der Umkehrfunktion}
	Sei $g $ eine stetige, streng monotone Funktion und $f $ die Umkehrfunktion. Wenn $g$ an der Stelle $y_* := f(x_*)$ differenzierbar ist mit $g′(y_*) \neq 0$, dann ist $f$ bei $x_*$ differenzierbar mit $f'(x_*)=\frac{1}{g'(y_*)}=\frac{1}{g'(f(x_*))}$

\end{sectionbox}

\begin{sectionbox}
	\subsection{Mittelwertsatz der Differentialrechnung}
	Ist $f : [a, b] \rightarrow \R$ eine stetige, auf $(a, b)$ differenzierbare Funktion. Dann existiert ein $\xi \in  (a, b)$ mit \\
	$f'(\xi)= \frac{f(b)-f(a)}{b-a}$
\end{sectionbox}

\begin{sectionbox}
	\subsection{Monotonie von Funktionen}
	Sei $f : [a, b] \rightarrow \R$ stetig und auf $(a, b)$ differenzierbar. $\forall x \in (a, b)$ gilt:
	\begin{itemize}
		\item $f$ ist monoton steigend genau dann, wenn $f'(x)\ge0$ (für streng >)
		\item $f$ ist monoton fallend genau dann, wenn $f'(x) \le 0$ (für streng <)
		\item $f$ ist konstant genau dann, wenn $f'(x) = 0$
	\end{itemize}

\end{sectionbox}

\begin{sectionbox}
	\subsection{Regel von l'Hôpital}
	Seien $a, b \in \R \cup \{-\infty , \infty \}$. Und $f, g : (a, b) \rightarrow \R$ zwei differenzierbare Funktionen. Sei $g′(x) \neq 0$ für alle $x \in [a,b]$, und es existiere der Limes:\\
	$\lim \limits_{x \to a} \frac{f'(x)}{g'(x)}=: c \in \R $
	\begin{enumerate}
		\item Falls $\lim\limits_{x \to a} f (x) = \lim \limits_{x \to a} g(x) = 0$ gilt: $\lim\limits_{x \to a} \frac{f(x)}{g(x)}= c$
		\item Falls $\lim\limits_{x \to a} f (x) = \lim \limits_{x \to a} g(x) = \pm \infty$ gilt: $\lim\limits_{x \to a} \frac{f(x)}{g(x)}= c$
	\end{enumerate}

\end{sectionbox}

%\begin{sectionbox}
%	\subsection{Bekannte Ableitungen}

%\end{sectionbox}


\begin{sectionbox}
	\subsection{Satz 8.28}
	Es sei $(f_n)_{n \in \N}$ eine Folge differenzierbarer Funktionen $f_n : [a, b] \rightarrow \C$. Wir nehmen an, dass
	\begin{itemize}
		\item die Funktionenfolge $(f'_n)_{n \in \N}$ der Ableitungen $f'_n : [a, b] \rightarrow \C$ gleichmäßig gegen ein $ g : [a, b] \rightarrow \C$ konvergiert
		\item die Zahlenfolge $(f_n(\overline{x}))_{n \in \N}$ für mindestens ein $\overline{x} \in  [a, b]$ konvergiert
	\end{itemize}
	Dann konvergiert die Funktionenfolge $(f_n)_{n \in \N}$ gleichmäßig gegen eine differenzierbare Funktion $f : [a,b] \rightarrow \C$, und es gilt $f' = g$. Ist zusätzlich jede Funktion $f_n$ stetig differenzierbar, so ist auch $f$ stetig differenzierbar.

\end{sectionbox}

\begin{sectionbox}
	\subsection{Taylor-Polynom}
	Taylor-Polynom für $f \in C^n(I)$, Grad $m \in \N$ und $m \le n$, an der Entwicklungsstelle $y \in I$:\\
	$T_m^f(y;x)= \sum \limits_{k=0}^{m} \frac{f^{(k)}(y)}{k!}(x-y)^k$

\end{sectionbox}

\begin{sectionbox}
	\subsection{Restgliedformel nach Lagrange}
	Es seien $f \in C^{m+1}([a, b]; \R)$ und $x \in [a, b]$ gegeben. Dann existiert zu jedem $x \in [a,b]$ mit $x \neq y$ ein $\xi \in (a,b)$ ”echt zwischen“ $y$ und $x$ so dass\\
	$f(x)=T_m^f(y;x)+\frac{f^{(m+1)}(\xi)}{(m+1)!}(x-y)^{(m+1)}$

\end{sectionbox}


\section{Integralrechnung}

\begin{sectionbox}
	\subsection{Jede stetige Funktion ist eine Regelfunktion }
	Eine Folge von Treppenfunktionen $(\phi_n)_{n \in \N}$, die gleichmäßig gegen $f$ konvergiert, ist gegeben durch
	$\phi_n(x) = f(x_k^{(n)})$  für alle $x \in (x_{k-1}^{(n)} ,x_k^{(n)}]$, sowie $\phi_n(a) = f(a)$, wobei $(x_k^{(n)})_{k=0}^{n}$  mit $x_k^{(n)}=a+(b-a) \frac{k}{n}$ eine Zerlegung von $[a,b]$ ist.
\end{sectionbox}

\begin{sectionbox}
	\subsection{Rechenregel für Integrale}
	Es seien $f, g : [a, b] \rightarrow \C$ Regelfunktionen und $\lambda, \mu \in \C$. Dann gilt:
	\begin{enumerate}
		\item Auch $\lambda f + \mu g : [a, b] \rightarrow \C$ ist eine Regelfunktion, und das Integral ist linear:\\ $\int \limits_a^b (\lambda f +\mu g)(x)dx= \lambda \int \limits_a^b f(x) dx + \mu \int \limits_a^b g(x)dx$
		\item Auch $|f| : [a, b] \rightarrow \R$ ist eine Regelfunktion:\\
		  	$|\int \limits_a^b f(x)dx | \le \int \limits_a^b |f(x)|dx \le \sup\limits_{a \le x \le b}|f(x)|$
		\item Sind $f, g$ reellwertig mit $f \le g$ dann:\\ $\int \limits_a^b f(x)dx \le \int \limits_a^b g(x)dx$
	\end{enumerate}
\end{sectionbox}
\begin{sectionbox}
	\subsection{Hauptsatz der Differential und Integralrechnung}

	Es sei $f : [a,b] \rightarrow \C$ eine stetige Funktion. Zu gegebenem $a \in [\alpha, \beta]$ definieren wir die Funktion $F : [a, b] \rightarrow \C$ durch:
	\begin{itemize}
		\item $F(x)= \int \limits_a^x f(t)dt $
		\item $\int \limits_a^b f(t)dt = F(b)-F(a)$
	\end{itemize}
	Es gilt $F'(x)=f(x)$


\end{sectionbox}

\begin{sectionbox}
	\subsection{Partielle Integration}
	Es seien $f,g : [a,b] \rightarrow \C$ zwei stetig differenzierbare Funktionen. Dann gilt:\\
$	\int \limits_a^b f'(x)g(x)dx = \left.f(x)g(x) \right|_a^b - \int \limits_a^b f(x)g'(x)dx$

\end{sectionbox}

\begin{sectionbox}
	\subsection{Substitutionsregel}
	Es sei $f : [a,b] \rightarrow \C$ eine stetige Funktion mit Stammfunktion $F : [a,b] \rightarrow \C$. Weiter sei $g : [\alpha,\beta] \rightarrow [a,b]$ eine stetig differenzierbare Funktion.\\
	$\int \limits_\alpha^\beta f(g(t))g'(t)dt = \int \limits_{g(\alpha)}^{g(\beta)} f(x)dx = \left.F(x) \right|_{g(\alpha)}^{g(\beta)} $



\end{sectionbox}

\begin{sectionbox}
	\subsection{Uneigentliches Integral}
	Ist $f : [a,b) \rightarrow \C$ mit $b \in  \R \cup \{+ \infty\}$(für $-\infty$ analog) eine uneigentliche Regelfunktion,  dann ist folgender Limes das uneigentliche Integral wenn er existiert :\\
	$\int \limits_a^b f(x)dx := \lim \limits_{c \to b} \int \limits_a^c f(x)dx$


\end{sectionbox}




\section{Bekannte Ableitungen/Stammfunktionen}
	\begin{sectionbox}
		\subsection{Ableitungen}
		\begin{itemize}
			\item $sin'(x)=cos(x), arcsin'(x)=\frac{1}{\sqrt{1-y^2}}$
			\item  $cos'(x)=-sin(x), arccos'(x)=-\frac{1}{\sqrt{1-y^2}}$

			\item $tan(x):=\frac{sin(x)}{cos(x)}$, $tan'(x)=1+(tan(x))^2=\frac{1}{(cos(x))^2}$
			\item $arctan'(x)=\frac{1}{1+y^2}$
		\end{itemize}
		\subsection{Stammfunktionen}
		\begin{itemize}
			\item $\int tan(x)dx=-ln(|cos(x)|)$
			\item $\int ln(x)dx=ln(x)x-x$
		\end{itemize}
	\end{sectionbox}


    \subsection{Elementarfunktionen}
\begin{sectionbox}
    
	\begin{itemize}\itemsep-1pt
		\item Exponentialfunktion\\
		$e^x = \sum\limits_{k = 0}^m\frac{x^k}{k!}$
		\item Trigonometrische Funktionen\\
		$\sin{x} = \sum\limits_{k = 0}^m(-1)^k\frac{x^{2k + 1}}{(2k + 1)!}$\\
		$\cos{x} = \sum\limits_{k = 0}^m(-1)^k\frac{x^{2k}}{(2k)!}$
		\item Logarithmusfunktion\\
		$\ln{(1 + x)} = \sum\limits_{k = 1}^m\frac{(-1)^{k + 1}}{k}x^k$
	\end{itemize}
\end{sectionbox}

\section{Ableitung und Integral}
\begin{sectionbox}
 
    $f$ diffbar, falls $f$ stetig und $\underset{h\rightarrow 0}{\lim}\frac{f(x_0+h)-f(x_0)}{h}=f'(x_0)$ exist.
    \subsubsection{Ableitungsregeln:}
    Linearität: $(\lambda f + \mu g)' (x) = \lambda f'(x) + \mu g'(x)$ \quad $\forall \lambda, \mu \in \mathbb R$ \\
    Produktregel: $(f \cdot g)' = f' g + f g'$\\
    Quotientenregel $\enbrace{\frac{f}{g}}' = \frac{f'g - fg'}{g^2}$\\
    Kettenregel: $\left( f(g(x)) \right)' = f'(g(x)) g'(x)$\\
    Potenzreihe: $f: ] \underbrace{-R+a, a+R}_{\subseteq D}	 [ \rightarrow \mathbb R, f(x) = \sum_{n=0}^{\infty} a_n (x -a)^n$ \quad $\Rightarrow$ \quad $f'(x) = \sum_{n=0}^{\infty} n a_{n} (x-a)^{n-1}$\\
    \textbf{Tangentengleichung:} $y=f(x_0)+f'(x_0)(x-x_0)$
    
    \subsubsection{Newton-Verfahren:}
    $x_{n+1}=x_n-\frac{f(x_n)}{f'(x_n)}$ mit Startwert $x_0$
    
    \subsubsection{Integrationsmethoden:}
    \begin{itemize}\itemsep0pt
    \item Anstarren + Göttliche Eingebung
    \item Partielle Integration: $\int uv'=uv-\int u'v$
    \item Substitution: $\int f(\underbrace {g(x)}_{t}) \underbrace {g'(x)\,\mathrm dx}_{\mathrm dt}=\int f(t)\, \mathrm dt$
    \item Logarithmische Integration: $\int \frac{g'(x)}{g(x)}dx=\ln\abs{g(x)}$
    \item Integration von Potenzreihen: $f(x)=\sum_{k=0}^{\infty}a_k(x-a)^k$ \\
    Stammfunktion: $F(x)=\sum_{k=0}^{\infty}\frac{a_k}{k+1}(x-a)^{k+1}$
    \item Brechstange: $t=\tan(\frac{x}{2})$ \quad $\mathrm dx = \frac{2}{1+t^2} \mathrm dt$ \\ $\sin(x) \rightarrow \frac{2t}{1+t^2}$ \qquad $\cos(x) \rightarrow \frac{1-t^2}{1+t^2}$
    \end{itemize}
    
    \subsubsection{Integrationsregeln}
    $\int_a^b f(x) \mathrm dx = F(b) - F(a)$\\
    $\int\lambda f(x)+\mu g(x) \, \mathrm dx=\lambda\int f(x) \, \mathrm dx + \mu\int g(x) \, \mathrm dx$
    
    \everymath{\displaystyle}	% Formeln ab hier groß Schreiben
    \begin{math}\renewcommand{\arraystretch}{1.8}
    \begin{array}{c|c|c}
    F(x) & f(x) & f'(x) \\ \hline
    \frac{1}{q+1}x^{q+1} & x^q & qx^{q-1} \\
    \raisebox{-0.2em}{$\frac{2\sqrt{ax^3}}{3}$} & \sqrt{ax} & \raisebox{0.2em}{$\frac{a}{2\sqrt{ax}}$}\\
    x\ln(ax) -x & \ln(ax) & \textstyle \frac{1}{x}\\
    e^x & e^x & e^x \\
    \frac{a^x}{\ln(a)} & a^x & a^x \ln(a) \\
    -\cos(x) & \sin(x) & \cos(x)\\
    \sin(x) & \cos(x) & -\sin(x)\\
    -\ln |\cos(x)| & \tan(x) & \frac{1}{\cos^2(x)} \\
    \ln |\sin(x)| & \cot(x) & \frac{-1}{\sin^2(x)} \\
    x\arcsin (x)+\sqrt{1-x^2} & \arcsin(x) & \frac{1}{\sqrt{1-x^2}}\\
    x\arccos (x)-\sqrt{1-x^2} & \arccos(x) & -\frac{1}{\sqrt{1-x^2}}\\
    x\arctan (x)-\frac{1}{2} \ln \left| 1+ x^2 \right| & \arctan (x) & \frac{1}{1+x^2} \\
    x\arccot (x)+\frac{1}{2} \ln \left| 1+ x^2 \right| & \arccot (x) & -\frac{1}{1+x^2} \\
    x \sinh ^{-1} (x) - \sqrt{x^2+1} & \sinh ^{-1} (x) & \frac{1}{\sqrt{x^2+1}}\\
    x \cosh ^{-1} (x) - \sqrt{x^2-1}  & \cosh ^{-1} (x) & \dfrac{1}{\sqrt{x^2-1}}\\
    \frac{1}{2}\ln(1-x^2) + x \tanh ^{-1} (x) & \tanh ^{-1} (x) & \frac{1}{1-x^2}\\
    \sinh(x) & \cosh(x) & \sinh (x) \\
    \cosh(x) & \sinh(x) & \cosh (x)\\
    \end{array}
    \end{math}
    \everymath{\textstyle}
    
    
    \subsubsection{Rotationskörper}
    Volumen: $V = \pi \int_a^b f(x)^2 \mathrm dx$\\
    Oberfläche: $O = 2 \pi \int_a^b f(x) \sqrt{1 + f'(x)^2} \mathrm dx$
    
    \subsubsection{Uneigentliche Integrale}
    $\int\limits_{\text{ok}}^{\text{böse}} f(x) \mathrm dx = \lim\limits_{b\rightarrow \text{böse}}\ \int\limits_{\text{ok}}^b f(x) \mathrm dx$\\ \\ \\
    Majoranten-Kriterium: $|f(x)|\le g(x) = \frac1{x^\alpha}$\\ \\
    $\int\limits_{1}^{\infty} \frac{1}{x^\alpha} \mathrm dx \begin{cases} \frac{1}{\alpha -1}, \quad \alpha > 1 \\ \infty, \qquad \alpha \le 1 \end{cases}$ \qquad
    $\int\limits_{0}^{1} \frac{1}{x^\alpha} \mathrm dx \begin{cases} \frac{1}{\alpha -1}, \quad \alpha < 1 \\ \infty, \qquad \alpha \ge 1 \end{cases}$\\ \\
    \textbf{Cauchy-Hauptwert}
    \begin{eqnarray*}
    & \text{CHW }\int\limits_{-\infty}^{\infty} f(x) \mathrm dx = \lim\limits_{b\rightarrow\infty} \int\limits_{-b}^b f(x) \mathrm dx \\
    & \text{CHW }\int\limits_{a}^{b} f(x) \mathrm dx=\lim\limits_{\varepsilon \rightarrow 0^+}\enbrace{\int\limits_{a}^{c-\varepsilon} f(x) \mathrm dx+\int\limits_{c+\varepsilon}^{b}f(x)\mathrm dx}
    \end{eqnarray*}
    
    \subsubsection{Laplace-Transformation von $f: [0,\infty[ \ra \mathbb R,\ s \mapsto f(s)$}
    $\mathcal L \; f(s) = F(s) = \int\limits_{0}^{\infty} e^{-st} f(t)\ \mathrm dt = \lim\limits_{b \ra \infty} \int\limits_{0}^{b} e^{-st} f(t)\ \mathrm dt$
    
    \subsubsection{Integration rationale Funktionen}
    Gegeben: $\int \frac{A(x)}{Q(x)} \mathrm dx \qquad A(x),Q(x)\in \mathbb R[x]$
    \begin{enumerate}\itemsep0pt
    \item Falls, $\deg A(x) \ge \deg Q(x) \Ra$ Polynomdivision: \\ $\frac{A(x)}{Q(x)} = P(x) + \frac{B(x)}{Q(x)}$ mit $\deg B(x) < \deg Q(x)$
    \item Zerlege $Q(x)$ in unzerlegbare Polynome
    \item Partialbruchzerlegung $\frac{B(x)}{Q(x)} = \frac{\ldots}{(x - a_n)} + \ldots + \frac{\ldots}{\ldots}$
    \item Integriere die Summanden mit folgenden Funktionen
    \end{enumerate}
    
    $\text{mit} ~ \lambda=x^2+px+q, ~~ \beta=4q-p^2 ~~ \text{und} ~p^2<4q$!
    $\int\frac{1}{(x-a)^m}\mathrm dx \begin{cases} \ln\left|x-a\right|, & m=1\\ \\ \frac{-1}{(m-1)(x-a)^{m-1}} &m\geq2 \end{cases}$\\ \\ \\
    $\int\frac{1}{(\lambda)^m} \mathrm dx \begin{cases} \frac{2}{\sqrt{\beta}} \arctan\frac{2x+p}{\sqrt{\beta}}, &m=1\\ \\ \frac{2x+p}{(m-1)(\beta)(\lambda)^{m-1}}+\frac{2(2m-3)}{(m-1)(\beta)} \int\frac{\mathrm dx}{(\lambda)^{m-1}}, &m\geq2 \end{cases}$\\ \\ \\
    $\int\frac{Bx+C}{(\lambda)^m} \mathrm dx \begin{cases} \frac{B}{2} \ln(\lambda) + (C-\frac{Bp}{2}) \int\frac{\mathrm dx}{\lambda}, &m=1\\ \\ \frac{-B}{2(m-1)(\lambda)^{m-1}} + (C-\frac{Bp}{2}) \int\frac{\mathrm dx}{(\lambda)^{m-1}}, &m\geq2 \end{cases}$\\ \\ \\
    Häufige Integrale nach Partialbruchzerlegung
    \begin{align*}
    &\int\frac{1}{x}\mathrm dx = \ln\abs{x}&&\int\frac{1}{x^2}\mathrm dx = -\frac{1}{x} \\
    &\int\frac{1}{a+x}\mathrm dx = \ln\abs{a+x} && \int\frac{1}{(a+x)^2}\mathrm dx = -\frac{1}{a+x} \\
    &\int\frac{1}{a-x}\mathrm dx = -\ln\abs{a-x} && \int\frac{1}{(a-x)^2}\mathrm dx = \frac{1}{a-x}
    \end{align*}
    \subsubsection{Paratialbruchzerlegung}
    \begin{equation*}
    \frac{B(x)}{Q(x)} = \frac{\ldots}{(x - x_0)} + \ldots + \frac{\ldots}{\ldots}
    \end{equation*}
    \paragraph{Ansatz}
    \begin{itemize}\itemsep0pt
    \item $n$-fache reelle Nullstelle $x_0$: $\frac{A}{x-x_0}+\frac{B}{(x-x_0)^2} +\dots$
    \item $n$-fache komplexe Nullstelle: $\frac{Ax+B}{x^2+px+q}+\frac{Ax+B}{(x^2+px+q)^2}$
    \end{itemize}
    \paragraph{Berechnung von $A,B,C,\dots$}
    \begin{itemize}\itemsep0pt
    \item Nullstellen in $x$ einsetzen (Terme fallen weg)
    \item Ausmultiplizieren und Koeffizientenvergleich
    \end{itemize}
    
    
\end{sectionbox}

% ======================================================================
% End
% ======================================================================
\end{document}