% Mathe Formelsammlung für HM1 SoSe 2011
% 2 Seiten

% Dokumenteinstellungen
% ======================================================================	

% Dokumentklasse (Schriftgröße 6, DIN A4, Artikel)
\documentclass[6pt,a4paper]{scrartcl}

% Pakete laden
\usepackage[utf8]{inputenc}		% Zeichenkodierung: UTF-8 (für Umlaute)   
\usepackage[german]{babel}		% Deutsche Sprache
\usepackage{multicol}			% Spaltenpaket
\usepackage{amsmath}
\usepackage{amssymb}
\usepackage{esint}				% erweiterte Integralsymbole
\usepackage{multicol}			% ermöglicht Seitenspalten  
\usepackage{wasysym}			% Blitz
\usepackage{graphicx}
\usepackage[colorlinks=true, urlcolor=black]{hyperref}
  
% Seitenlayout und Ränder:
\usepackage{geometry}
\geometry{a4paper,landscape, left=6mm,right=6mm, top=0mm, bottom=3mm,includeheadfoot} 

%Kopf- und Fußzeile
\usepackage{fancyhdr}
\pagestyle{fancy}
\fancyhf{}

   \fancyfoot[C]{\textbf{Lineare Algebra} von Lukas Kompatscher (\href{mailto:lukas.kompatscher@tum.de}{lukas.kompatscher@tum.de})}
   \renewcommand{\headrulewidth}{0.0pt} %obere Linie ausblenden
   \renewcommand{\footrulewidth}{0.1pt} %untere Linie

   \fancyfoot[R]{Stand: \today \qquad \thepage}
   \fancyfoot[L]{Homepage: www.latex4ei.de - Fehler bitte sofort melden.}
	
% Schriftart SANS für bessere Lesbarkeit bei kleiner Schrift
\renewcommand{\familydefault}{\sfdefault} 


% Custom Commands
\renewcommand{\thesubsection}{\arabic{subsection}}
\newcommand{\me}[1]{\ensuremath{\left\{#1\right\}}}
\newcommand{\dme}[2]{\ensuremath{\left\{#1\,\vert\,#2 \right\}}}
\newcommand{\abs}[1]{\ensuremath{\left\vert#1\right\vert}}
\newcommand{\un}[1]{\; \unit{#1} }
\newcommand{\unf}[2]{\;\left[ \unitfrac{#1}{#2} \right]}
\newcommand{\norm}[2][\relax]{\ifx#1\relax \ensuremath{\left\Vert#2\right\Vert}\else \ensuremath{\left\Vert#2\right\Vert_{#1}}\fi}
\newcommand{\enbrace}[1]{\ensuremath{\left(#1\right)}}
\newcommand{\nira}[1]{\ensuremath{\overset{n \rightarrow \infty}{\longrightarrow}}}
\newcommand{\os}[2]{\ensuremath{\overset{#1}{#2}}}
\makeatletter
\newcommand{\Ra}[0]{\ensuremath{\Rightarrow}}
\newcommand{\ra}[0]{\ensuremath{\rightarrow}}
\newcommand{\gk}[1]{\ensuremath{\left\lfloor#1\right\rfloor}}
\newcommand{\sprod}[2]{\ensuremath{%
  \setbox0=\hbox{\ensuremath{#2}}
  \dimen@\ht0
  \advance\dimen@ by \dp0
  \left\langle #1\rule[-\dp0]{0pt}{\dimen@},#2\right\rangle}}
  
%Custom functions
\DeclareMathOperator{\arccot}{arccot}
\DeclareMathOperator{\im}{im}
\DeclareMathOperator{\diag}{diag}
\DeclareMathOperator{\rang}{rang}
\DeclareMathOperator{\col}{col}
\DeclareMathOperator{\row}{row}
\DeclareMathOperator{\Kern}{kern}
\DeclareMathOperator{\proj}{proj}



% Dokumentbeginn
% ======================================================================
\begin{document}
%\section{}
% ----------------------------------------------------------------------

% Aufteilung in Spalten
\begin{multicols*}{4}
\parbox{2.3cm}{
	
}
\parbox{4cm}{
	\huge{\textbf{Lineare Algebra}}
}
\subsection{Allgemeines} % (fold)
\label{sub:allgemeines}

Dreiecksungleichung \qquad \qquad \qquad
\begin{math}\begin{array}{l}
	\abs{x + y} \le \abs{x} + \abs{y}
\end{array}\end{math} \\
Cauchy-Schwarz-Ungleichung: \qquad 
\begin{math}\begin{array}{l}
\left| \sprod{x}{y} \right| \le \| x\| \cdot \| y\|
\end{array}\end{math}\\
$\mathbb{K}$ steht für $\mathbb{R}$ und $\mathbb{C}$\\
$\mathbb{I}_n$ ist die $nxn$-Einheitsmatrix.

\subsection{Matrizen}
% ----------------------------------------------------------------------
Die Matrix $A=(a_{ij}) \in \mathbb K^{m\times n}$ hat $m$ Zeilen mit Index $i$ und $n$ Spalten mit Index $j$.

\subsubsection{Allgemeine Rechenregeln}
\textbf{Merke:} Zeile vor Spalte! (Multiplikation, Indexreihenfolge, etc.)\\

\begin{tabular}{ll}	
	1)  $A+0=A$ & 2)  $1 \cdot A=A$ \\
	3)  $A+B=B+A$ & 4) $A \cdot B \ne B \cdot A$ (im Allg.) \\
	5)  $(A+B)+C=A+(B+C)$ & 6) $\lambda (A+B) = \lambda A + \lambda B$\\ 
\end{tabular}
Multiplikation von $A\in \mathbb K^{m\times r}$ und $B\in \mathbb K^{r\times n}$: $AB\in\mathbb K^{m\times n}$

\subsubsection{Elementare Zeilenumformungen (EZF) (gilt äquiv. für Spalten)}
$A \in \mathbb K^{m\times n}$ hat $m$ Zeilen $z_i\in \mathbb K^n$
\begin{itemize}\itemsep0pt
\item Vertauschen von Zeilen
\item Multiplikation einer Zeile mit $\lambda\ne 0$ 
\item Addition des $\lambda$-fachen der Zeile $z_i$ zur Zeile $z_j$
\end{itemize}

\subsubsection{Transponieren}
$A=(a_{ij})\ \in \mathbb K^{m\times n}$ gilt: $A^\top=(a_{ji})\ \in \mathbb K^{n\times m}$\\
\textbf{Regeln:}\\
$(A+B)^\top=A^\top+B^\top$\qquad $(A\cdot B)^\top=B^\top\cdot A^\top$\qquad \\
$(\lambda A)^\top=\lambda A^\top$ \qquad \qquad \qquad $(A^\top)^\top=A$\\
\\
$A\in \mathbb K^{n\times n}$ ist symmetrisch, falls $A=A^\top$\qquad ($\Rightarrow$ diagbar)\\
$A\in \mathbb K^{n\times n}$ ist schiefsymmetrisch, falls $A=-A^\top$\\
$A\in \mathbb K^{n\times n}$ ist orthogonal (Spalten-/Zeilenvektoren=ONB), falls:\\
$AA^\top=\mathbb{I}_n \quad \Leftrightarrow \quad A^\top=A^{-1} \quad \Leftrightarrow \quad \det A=\pm 1$\\
$A\in \mathbb C^{n\times n}$ ist hermitesch, falls $A=\overline{A}^\top$  \quad (kmplx. konj. u. transp.)


\subsubsection{Inverse Matrix von $A\in \mathbb K^{n\times n}$}
Für die inverse Matrix $A^{-1}$ von $A$ gilt: $A^{-1}A=\mathbb{I}_n$\\
$(A^{-1})^{-1}=A$ \qquad $(AB)^{-1}=B^{-1}A^{-1}$ \\
$(A^\top)^{-1}=(A^{-1})^\top$\\
\\
$A\ \in \mathbb K^{n\times n}$ ist invertierbar, falls: $\det (A) \ne 0 \quad \lor \quad \rang(A)=n$\\
\\
Berechnen von $A^{-1}$ nach Gauß:\\
$AA^{-1}=\mathbb{I}_n\quad\Rightarrow\quad (A|\mathbb{I}_n)\overset{EZF}{\longrightarrow}(\mathbb{I}_n|A^{-1})$\\
2x2-Matrix:$\enbrace{\begin{matrix}
a & b \\
c & d
\end{matrix}}^{-1} = \frac{1}{ad-bc}\begin{pmatrix}
d & -b \\
-c & a
\end{pmatrix}$

\subsubsection{Rang einer Matrix $A\in \mathbb K^{m\times n}$}
{\tiny (N0-Zeilen = Nicht-Null-Zeilen)}\\ \\
\textbf{Bringe A auf ZSF} \\
Rang (Zeilrang) $\rang(A)$: Anzahl N0-Zeilen \\     
Zeilenraum $\row(A)$: Erzeugnis der Zeilen, $\text{Basis}(\row(A)) = \{\text{ N0-Zeilen }\}$ \\
Kern: $\Kern(A) = \dme{x \in \mathbb K^n}{Ax= 0}$ \\
Dimensionsformel: $\rang(A) + \mathrm{dim}(\Kern(A)) = n$ \\
\textbf{Bringe A auf Spaltenstufenform (transponieren, ZSF)} \\
Spaltenrang: Anzahl der N0-Spalten\\
Spaltenraum $\col(A)$: Erzeugnis der Spalten, $\text{Basis}(\col(A)) = \{\text{ N0-Spalten }\}$ \\
Bild = Spaltenraum: Erzeugnis der Spalten 
\subsubsection{Matrixpotenzen}
Gegeben: $A \in\mathbb{R}^{mxn}, x \in\mathbb{R}^n$.\\
Gesucht: Lösung von $A^n$.\\
\begin{itemize}
	\item Bestimme Eigenwerte $\lambda$ und Eigenvektoren $v$ von $A$.
	\item Bestimme $\alpha_1, ..., \alpha_k$ mit $x = \alpha_1v_1 + ... + \alpha_kv_k$.
	\item $A^nx = \alpha_1\lambda_1^nv_1+ ... + \alpha_k\lambda_k^nv_k$.
\end{itemize}
Gegeben: $A \in\mathbb{R}^{nxn}, A$ diagonalisierbar.\\
Gesucht: $A^n$.
\begin{itemize}
	\item $A^n = SD^nS^{-1}$ mit $D^n = \begin{pmatrix}
	\lambda_1^n & \ldots & 0 \\
	\vdots & \ddots & \vdots\\
	0 & \ldots & \lambda_k^n \\
	\end{pmatrix}$.
\end{itemize}

\subsubsection{Lineares Gleichungssystem LGS}
Das LGS $Ax=b$ kurz $(A|b)$ mit $A\in \mathbb K^{m\times n}$, $x\in \mathbb K^n$, $b\in \mathbb K^m$ hat $m$ Gleichungen und $n$ Unbekannte.\\
\\
\textbf{Lösbarkeitskriterium:}\\
Ein LGS $(A|b)$ ist genau dann lösbar, wenn: $\rang(A)=\rang(A|b)$\\
Die Lösung des LGS $(A|b)$ hat $\dim(\Kern(A)) = n-\rang(A)$ frei wählbare Parameter.\\
\\
Das LGS hat eine Lsg. wenn $\det A \not= 0$ \quad $\rightarrow \exists A^{-1}$ \\
Das homogene LGS: $(A|0)$ hat stets die triviale Lösung $0$\\
Summen und Vielfache der Lösungen von $(A|0)$ sind wieder Lösungen.

\subsubsection{Determinante von $A\in \mathbb K^{n\times n}$: $\det(A)=|A|$}

\begin{itemize}\itemsep0pt
\item $|A|=\sum\limits_{i=1}^n (-1)^{i+j} \cdot a_{ij} \cdot |A_{ij}|$ \qquad Entwicklung n. $j$-ter Spalte
\item $|A|=\sum\limits_{j=1}^n (-1)^{i+j} \cdot a_{ij} \cdot |A_{ij}|$ \qquad Entwicklung n. $i$-ter Zeile
\item $\det\begin{pmatrix}A&0\\C&D\end{pmatrix}=\det\begin{pmatrix}A&B\\0&D\end{pmatrix}=\det(A)\cdot\det(D)$
\item $\begin{vmatrix}\lambda_1&&* \\ &\ddots& \\ 0&&\lambda_n \end{vmatrix} = \lambda_1\cdot \ldots\cdot \lambda_n = \begin{vmatrix} \lambda_1&&0  \\  &\ddots& \\  *&&\lambda_n \end{vmatrix}$
\item $A=B \cdot C \quad \Rightarrow \quad |A|=|B| \cdot |C|$
\item $\det(A)=\det(A^\top)$
\item Hat $A$ zwei gleiche Zeilen/Spalten $\Rightarrow |A|=0$
\item $\det(\lambda A)=\lambda^n \det(A)$
\item Ist $A$ invertierbar, so gilt: $\det(A^{-1})=(\det(A))^{-1}$
\item $\det(AB) = \det(A) \det(B) = \det(B) \det(A) = \det(BA)$
\end{itemize}
\textbf{Umformung Determinante}
\begin{itemize}\itemsep0pt
\item Vertauschen von Zeilen/Spalten ändert Vorzeichen von $|A|$
\item Zeile/Spalte mit $\lambda$ multiplizieren, $|A|$ um Faktor $\lambda$ größer
\item Addition des $\lambda$-fachen der Zeile X zur Zeile Y ändert $|A|$ nicht 
\end{itemize}
\textbf{Vereinfachung für Spezialfall $A\in \mathbb K^{2\times 2}$}\\
$A=\begin{pmatrix}a&b\\c&d\end{pmatrix} \Rightarrow \det(A)=|A|=ad-bc$

\subsubsection{Äquivalente Aussagen für $A\in \mathbb K^{n\times n}$}
\begin{tabular}{ll}
	1)  $A$ ist invertierbar & 2) $\dim(\col(A))=\dim(\row(A))=n$\\
	3)  $\Kern(A)={0}$ & 4) Die strenge ZSF von $A$ ist $\mathbb{I}_n$\\
	5) $\det(A)\ne0$ & 6) Zeilen/Spalten von $A$ linear unabhängig\\
	7) $Ax=b$ hat eine & 8) 0 ist kein Singulärwert von $A$\\\ \ \ \ eindeutige Lösung $\forall b\in\mathbb{R}^n$ & 9) Lineare Abbildung ist bijektiv\\
	10)  $\rang(A)=n$ & 11) 0 ist kein Eigenwert von $A$\\

\end{tabular}

\subsection{Vektoren}
Ein Vektor ist ein $n$-Tupel reeller oder komplexer Zahlen, also ein Element aus dem $\mathbb K^n$.
\subsubsection{Skalarprodukt $\langle v,w \rangle: V\times V \rightarrow \mathbb R$} 
	\begin{enumerate}\itemsep0pt
	\item Linear: $\langle u,v+w \rangle=\langle u,v \rangle + \langle u,w \rangle \land \langle u,\lambda v \rangle = \lambda \langle u, v \rangle$ 
	\item Symmetrisch: $\langle v,w \rangle=\langle w,v \rangle$
	\item Positiv definit: $\langle v,v \rangle\ge0$  \qquad $\langle v,v \rangle=0 \Leftrightarrow v=0$
	\end{enumerate} 
\textbf{Kanonisches Skalarprodukt} \\
$\langle v,w \rangle=v^\top  w=v_1w_1+\dots+v_nw_n$\\ \\
\textbf{Skalarprodukt} bzgl. sym., quadr. und positiv definiter Matrix $A\in \mathbb K^{n\times n}$\\
$\langle v,w \rangle_A=v^\top A w$\\
\textbf{Skalarprodukt Polynome} $\sprod{p(x)}{q(x)}=\int\limits_{0}^{1}p(x)q(x)\,dx$ \\
\textbf{Orthogonalität} $\sprod{a}{b} = 0 \Leftrightarrow a\perp b$ \\
\textbf{Projektion} eines Vektor $v$ längs $a$: $\proj_a(v) = \frac{\sprod{v}{a} }{\sprod{a}{a} }\cdot a$\\
\textbf{Orthogonale Zerlegung} eine Vektors $v$ längs $a$:\\ 
$v = \proj_a(v) + \proj_{a^\perp}(v) \Rightarrow \proj_{a^\perp}(v) = v - \proj_a(v)$\\
\textbf{Winkel} \quad 	$\cos \phi = \frac{\sprod{a}{b}}{\norm{a} \norm{b}} $ \qquad
$\phi = \arccos \enbrace{ \frac{\sprod{a}{b} }{\norm{a} \norm{b} } }$

\subsubsection{Kreuzprodukt (Vektorprodukt)}
$a\times b=\left( \begin{matrix} a_2b_3-a_3b_2\\a_3b_1-a_1b_3\\a_1b_2-a_2b_1\end{matrix}\right)$\qquad $a,b\ \in \mathbb R^3$\\
\\
$a\times b \perp a,b$ \qquad (falls $a\times b=0\ \Leftrightarrow\ a,b$\ linear abhängig)\\
$a \times b = -b \times a$\\
$\norm{a\times b}=\norm{a}\cdot\norm{b}\cdot \sin\left(\measuredangle (a,b)\right)\mathrel{\widehat{=}}$\ Fläche des Parallelogramms\\
Graßmann-Identität: $a\times(b \times c)\equiv b\cdot(a \cdot c)-c\cdot(a \cdot b)$\\
\\
\textbf{Spatprodukt}\\
$[a,b,c]:=\langle  a\times b, c\rangle=\det (a,b,c)\mathrel{\widehat{=}}$\ Volumen des Spates.\\
$[a,b,c]>0\ \Leftrightarrow\ a,b,c$\ bilden Rechtssystem \\ $[a,b,c]=0\ \Leftrightarrow\ \{a,b,c\}$\ linear abhängig

\subsection{Vektorräume (VR)}
% --------------------------------------------------------------
Eine nichtleere Menge V mit zwei Verknüpfungen $+$ und $\cdot$ heißt $K$-Vektorraum über dem Körper $\mathbb K$. \\
\textbf{Bedingung $(u,v,w\in V \quad \lambda,\mu \in \mathbb{R})$}
\begin{enumerate}\itemsep0pt
\item $v+w\in V$ \qquad $\lambda v \in V$
\item $u+(v+w)=(u+v)+w$
\item $0\in V: v+0=v$
\item $v'\in V: v+v'=0$
\item $v+w=w+v$
\item $\lambda(v+w)=\lambda v + \lambda w$
\item $(\lambda + \mu)v=\lambda v +\mu v $
\item $(\lambda \mu)v = \lambda(\mu v)$
\item $1v=v$
\end{enumerate}\itemsep0pt
\subsubsection{Untervektorraum (UVR) $U\subset V (u,v\in U \quad \lambda\in\mathbb{R})$}
\begin{enumerate}\itemsep0pt
\item $U\ne \emptyset \qquad (0\in U)$
\item $u+v\in U$
\item $\lambda u \in U$
\end{enumerate}

\subsubsection{Basis (Jeder VR und jeder UVR besitzt eine Basis!)} % (fold)
\label{sub:basis}
 Eine Teilmenge $B\subset V$ heißt Basis von $V$, wenn gilt:
\begin{itemize}\itemsep0pt
	\item $\mathrm{span}(B) =V$, $B$ erzeugt $V$
	\item $B$ ist linear unabhängig
\end{itemize}    

\subsubsection{Dimension}
$n=\dim(V)=\abs{B} = $ Mächtigkeit von $B$\\
Mehr als $n$ Vektoren aus $V$ sind stets linear abhängig. \\
Für jeden UVR $U \subset V$ gilt: $\dim (U) \le \dim (V)$ 

\subsubsection{Linearkombination}
Jeder Vektor $v\in\mathbb{K}^n$ kann als Linearkombination einer Basis $B=\{b_1, \dots, b_n\} \subset \mathbb{K}^n$ dargestellt werden
\begin{equation*}
v=\lambda_1 b_1 + \dots + \lambda_n b_n \Rightarrow \text{Gauß} \ \Big(b_1 \ b_2 \ b_3 \ |\ v\ \Big)
\end{equation*}
\textbf{Linear Unabhängig:}
Vektoren heißen linear unabhängig, wenn aus: \\
$\lambda_1 v_1 + \dots + \lambda_n v_n = 0$ folgt, dass $\lambda_1 = \dots = \lambda_n = 0$

\subsubsection{Orthogonalität}
$B\subset V$ heißt
\begin{itemize}\itemsep0pt
\item \textbf{Orthogonalsystem}, wenn $\forall v,w\in B: v\perp w$
\item \textbf{Orthogonalbasis}, wenn $B$ Orthogonalsystem und Basis von $V$
\item \textbf{Orthonormalsystem}, wenn $B$ Orthogonalssystem u. $\forall v\in B: \norm{v}=1$
\item \textbf{Orthonormalbasis(ONB)},  wenn $B$ Orthonormalsystem u. Basis von V ist
\end{itemize}
\textbf{Matrix $A$ heißt orthogonal}, wenn $A^\top A = \mathbb{I}_n$
\begin{itemize}\itemsep0pt
\item $A^{-1}=A^\top $
\item $\det{A}=\pm1$
\item Spalten bilden ONB
\item Zeilen bilden ONB
\item $\norm{Av}=\norm{v}$
\end{itemize}\itemsep0pt
\textbf{Orthonormalisierungsvefahren einer Basis $\{v_1,\ldots,v_n\}$ nach Gram-Schmidt}
\begin{enumerate}\itemsep0pt
\item $b_1=\frac{v_1}{\|v_1\|}$ \qquad (Vektor mit vielen 0en oder 1en)
\item $b_{2}= \frac{c_2}{\norm{c_2}}$\ \ mit \ \ $c_2=v_2-\frac{\sprod{v_2}{v_1}}{\sprod{v_1}{v_1}}\cdot v_1$
\item $b_{3}= \frac{c_3}{\norm{c_3}}$\ \ mit \ \ $c_3=v_3-\frac{\sprod{v_3}{v_1}}{\sprod{v_1}{v_1}} \cdot v_1-\frac{\sprod{v_3}{c_2}}{\sprod{c_2}{c_2}}\cdot c_2$
\end{enumerate}
\textbf{Erweitern einer ONB von $V$ auf eine ONB des $\mathbb{R}^n$}
\begin{enumerate}\itemsep0pt
	\item Vektor $e_i \text{ mit } i \in \{1...n\}$ so wählen, sodass die Skalarprodukte möglichst einfach zu berechnen sind.
	\item Gram-Schmidt für $e_i$ $\Rightarrow$ Ergebnis zur Basis hinzufügen
	\item So lange Wiederholen bis die Basis $n$ Vektoren besitzt.
	\item Alle neu hinzugefügten Vektoren bilden zusammen eine ONB von $V^\perp$
\end{enumerate}
\textbf{Orthogonale Projektion auf UVR} \\
Gegeben: Vektorraum $V \in \mathbb{R}^n$, $v\in V$, Untervektorraum $U\subset V$
\begin{enumerate}\itemsep0pt
\item Basis von $U$ bestimmen
\item Orthogonalisiere Basis $\{u_1,u_2,u_3,\ldots\}$ von $U$
\item $\proj_U(v) = \frac{\sprod{v}{u_1}}{\sprod{u_1}{u_1}}u_1 + \frac{\sprod{v}{u_2}}{\sprod{u_2}{u_2}}u_2 +\dots$
\item $\proj_{U^\perp}(v)=v-\proj_U(v)$
\item Abstand von $v$ zu $U$ = $\norm{\proj_{U^\perp}(v)}$
\end{enumerate}
Alternative Methode
\begin{enumerate}\itemsep0pt
\item Basis $\{b_1,\ldots,b_r\}$ von $U$ bestimmen
\item Setze $A = \big(b_1\ b_2\ \dots \ b_r\big) \in \mathbb{R}^{n\times r}$
\item Löse das LGS $A^\top Ax=A^\top v$ und erhalte den Lösungsvektor $x=(\lambda_1, \dots,\lambda_r)^\top $
\item $\proj_U(v)=\lambda_1 b_1 +\dots +\lambda_r b_r$
\end{enumerate}

\subsection{Norm}
\textbf{Norm von Vektoren}
$\norm{a}=\sqrt{\sprod{a}{a}} =\sqrt{a_1^2+a_2^2+\ldots +a_n^2}$\\
$\forall v, w \in \mathbb{R}^n$:
\begin{enumerate}\itemsep0pt
\item $\norm{\lambda v}=\abs{\lambda}\norm v$
\item $\norm{v+w}\le \norm{v} + \norm{w}$
\end{enumerate}

\subsection{Lineare Abbildungen}
Abbildung $f:V\rightarrow W$ ist linear, wenn
\begin{enumerate}\itemsep0pt
\item $f(0)=0$
\item $f(a+b)=f(a)+f(b)$
\item $f(\lambda a)=\lambda f(a)$
\end{enumerate}
$\Rightarrow$ Abbildung als Matrix darstellbar (siehe Abbildungsmatrix)\\ \\
\textbf{Injektiv}, wenn aus $f(x_1)=f(x_2) \Rightarrow x_1=x_2$\\
\textbf{Surjektiv}: $\forall y\in W \ \exists x\in V:f(x)=y$\\ \quad (Alle Werte aus $W$ werden angenommen.)\\
\textbf{Bijektiv}(Eineindeutig): $f$ ist injektiv und surjektiv $\Rightarrow$ $f$ umkehrbar.
\subsubsection{Koordinatenvektor bezüglich einer Basis $B$}
Gegeben: Vektorraum $V \in\mathbb{R}^n$, $v \in V$.\\
Gesucht: $[v]_B$ (Koordinaten von $v$ bezüglich der Basis $B$).
\begin{enumerate}
	\item Bestimme Basis $B$ von $V$.
	\item Löse das LGS $Bx = v$.
	\item $[v]_B = x$.
\end{enumerate}
\subsubsection{Abbildungsmatrix (Darstellungsmatrix)}
Lineare Abbildung $f:\mathbb{R}^n \rightarrow \mathbb{R}^m$ \\
Abbildungsmatrix spaltenweise:
$[f] = \begin{pmatrix}f(e_1) & \dots & f(e_n)
\end{pmatrix} $ \\ \\
\textbf{Allgemein} $f:V\rightarrow W$ mit $V, W$ Vektorräume \\
$B=(b_1,\dots,b_n)$ ist eine Basis von $V$, $\exists B^{-1}$. \\
$[f]_B := B^{-1}[f]B$.\\ \\
Gesucht: $x \in\mathbb{R}^n$ mit $f(x) = b$ und $b\in\mathbb{R}^n$.\\
- Löse das LGS $[f]_Bx = B^{-1}b$.\\
Folgende Aussagen sind äquivalent:\\
a) $[f]_B = \begin{pmatrix}
\lambda_1 & & & \\
& \lambda_2 & & \\
& & \ddots & \\
& & & \lambda_n \\ 
\end{pmatrix}$.\\
b) $f(b_1) = \lambda_1b_1, ..., f(b_n) = \lambda_nb_n$.\\ \\

$C=(c_1,\dots,c_n)$ ist eine Basis von $V$. \\
$\Rightarrow [f]_B^C = \begin{pmatrix}
\vert & \vert &  & \vert \\
f(b_1)_C & f(b_2)_C & \cdots & f(b_n)_C\\
\vert & \vert &  & \vert
\end{pmatrix}$ \\
ist die Darstellungsmatrix von $f$ bzgl. $B$ und $C$. \\
''In der j-ten Spalte der Abbildungsmatrix stehen die Koordinaten des Bildes $f(b_j)$ bzgl. der Basis $C=(c_1,\dots,c_m)$'' \\ \\
\textbf{Eigenschaften von $f$ mit Hilfe von $[f]$}
\begin{itemize}\itemsep0pt
\item $\Kern(f) = \{ x\in\mathbb{R}^n : f(x) = 0 \}$
\item $\im(f) = \col([f])$
\item $f$ injektiv, wenn $\Kern([f])=\{0\}$
\item $f$ surjektiv, wenn $\im([f])=\mathbb{R}^m$
\item $f$ bijektiv, wenn $[f]$ invertierbar
\item $f$ ist bijektiv $\Leftrightarrow f$ ist injektiv $\Leftrightarrow f$ ist surjektiv
\end{itemize}
\subsubsection{Transformationsmatrix}
Transformationsmatrix der Koordinaten von $B$ zu $C$: $_CT_B$\\
\textbf{Regeln und Berechnung:}
\begin{itemize}
	\item $_{\mathbb{I}_n}T_B = T_B$: Vektoren der Basis $B$
	\item $_CT_B = {_CT} \cdot T_B$
	\item $(_CT_B)^{-1} = {_BT_C}$
	\item $[v]_C = {_CT_B}\cdot[v]_B$
	\item $C=B\cdot {_CT_B}$
\end{itemize}
\subsection{Diagonalisierung (Eigenwerte und Eigenvektoren)}
Gegeben: Quadratische Matrix $A \in \mathbb{R}^{n\times n}$.\\
Gilt $Av = \lambda v$ mit $v\ne 0$, so nennt man
\begin{itemize}\itemsep0pt
\item $v\in V$ einen \textbf{Eigenvektor} von $A$ zum \textbf{Eigenwert} $\lambda \in \mathbb{R}$ und 
\item $\lambda \in \mathbb{R}$ einen \textbf{Eigenwert} von A zum \textbf{Eigenvektor} $v\in V$
\end{itemize}
Ist $\lambda$ ein Eigenwert von A, so nennt man den Untervektorraum
\begin{itemize}\itemsep0pt
\item $\text{Eig}_A(\lambda) = \{v \in \mathbb{R}^n | Av = \lambda v\}$ den Eigenraum von A zum Eigenwert $\lambda$ und
\item $\dim({\text{Eig}_A(\lambda))}$ die geometrische Vielfachheit des Eigenwerts $\lambda$
\item $\text{geo}(\lambda) = \dim(\text{Eig}_A(\lambda))$
\end{itemize}
\textbf{Diagonalisieren von Matrizen}\\
$A$ ist diag.bar falls eine invertierbar Matrix $B$ existiert, sodass
\begin{equation*}
D=B^{-1}AB \ \Leftrightarrow \ A=BDB^{-1}
\end{equation*}
und $D$ eine Diagonalmatrix ist.
\begin{itemize}\itemsep0pt
\item Eine Matrix ist genau dann diagonalisierbar wenn $\text{alg}(\lambda)=\text{geo}(\lambda)$ für jeden Eigenwert $\lambda$ von $A$ gilt.
\item Jede Matrix $A \in \mathbb{R}^{n\times n}$ mit $n$ verschiedenen Eigenwerten ist diagonalisierbar.
\item Eine symmetrische Matrix hat nur reelle Eigenwerte und ist diagonalisierbar.
\item Die Determinante einer Matrix ist gleich dem Produkt der Eigenwerte: $\det(A)=\lambda_1\dots\lambda_n$
\end{itemize}
\subsubsection{Rezept: Diagonalisieren}
Gegeben: $A\in \mathbb{R}^{n\times n}$
\begin{enumerate}\itemsep0pt
\item Bestimme das charakteristische Polynom von $A$
\begin{equation*}
p_A(\lambda)=\det(A-\lambda \mathbb{I}_n)
\end{equation*}
\item Charakteristische Polynom $p_A$ in Linearfaktoren zerlegen.
\begin{equation*}
p_A(\lambda)=(\lambda_1-\lambda)^{\nu_1}\dots(\lambda_r-\lambda)^{\nu_r}
\end{equation*}
Es gilt $\nu_1 + \dots + \nu_r=n$ \\
$\lambda_1,\dots, \lambda_r$ sind die Eigenwerte mit algebraischer Vielfachheit $\text{alg}(\lambda_i)=\nu_i$\\
Ist $p_A$ nicht vollständig in Linearfaktoren zerlegbar $\Rightarrow$ A nicht diagonalisierbar!
\item Bestimme zu jeden Eigenwert $\lambda_i$ den Eigenraum $V_i$
\begin{equation*}
V_i=\Kern(A-\lambda_i\mathbb{I}_n)=\text{span}(B_i)
\end{equation*}
Die Vektoren der Basis $B_i$ sind die Eigenvektoren von $\lambda_i$.\\ \\
\textbf{Einfacher:} Der Eigenvektor $v_i$ ist Lösung des homogenen LGS
\begin{equation*}
(A-\lambda_i \mathbb{I}_n)v_i=0
\end{equation*}
$\dim(V_i)=\text{geo}(\lambda_i)$  geometr. Vielfachheit des Eigenwerts $\lambda_i$. \\
Gilt geo$(\lambda_i)\ne\text{alg}(\lambda_i)$ für ein $i$, ist A nicht diagonalisierbar!
\item $B=(v_1 \ \dots v_n)$ setzt sich aus den Eigenvektoren zusammen. \\
$D=\diag(\lambda_1,\dots,\lambda_n)$ ist die Diagonalmatrix der Eigenwerte.
\begin{equation*}
D=B^{-1}AB \ \Leftrightarrow \ A=BDB^{-1}
\end{equation*}
\end{enumerate}
\subsection{QR-Zerlegung}
$A = QR$, wobei $Q$ orthogonal und R oben dreieckig.\\
\textbf{Vorgehen}
\begin{itemize}\itemsep0pt
 \item $Q$ berechnen durch Gram-Schmidt mit den Spalten von $A$, beginnend bei der ersten
 \item Die Koeffizienten von $R$ ergeben sich durch Umstellen der jeweiligen Gram-Schmidt Gleichungen auf die Spalten von $A$
 \item Alternativ gilt: $R = Q^TA$
\end{itemize}

\subsection{Kleinstes-Quadrate-Problem}
Für $Ax = b$ lautet die
\textbf{Normalengleichung}
$A^TAx = A^Tb$\\
$\Rightarrow$ optimale Lösung mit minimalem quadratischen Fehler (existiert immer).

\subsection{Singulärwertzerlegung}
Bei der Singulärwertzerlegung wird eine beliebige Matrix $A\in \mathbb{R}^{m\times n}$ als Produkt dreier Matrizen $U$, $S$ und $V$ geschrieben
\begin{equation*}
A=USV^\top
\end{equation*}
mit $U\in \mathbb{R}^{m\times m}$, $S\in \mathbb{R}^{m\times n}$ und $V\in \mathbb{R}^{n\times n}$.\\
$U$ und $V$ sind orthogonal, $S$ ist eine Diagonalmatrix.
\subsubsection{Rezept: Singulärwertzerlegung}
Gegeben: $A\in \mathbb{R}^{m\times n}$
\begin{enumerate}\itemsep0pt
\item Bestimme alle Eigenwerte $\lambda_j$ und Eigenvektoren $v_j$ der Matrix $A^\top A\in \mathbb{R}^{n\times n}$ und ordne sie \\ $\lambda_1\ge\lambda_2\ge \dots \ge \lambda_r>\lambda_{r+1}=\dots=\lambda_n=0$ mit $r\le n$
\item Bestimme eine ONB des $\mathbb{R}^n$ aus den Eigenvektoren $v_j$ und erhalte $V=\begin{pmatrix} 
v_1 &\dots & v_n
\end{pmatrix} \in \mathbb{R}^{n\times n}$
\item Die Singulärwerte sind $\sigma_j=\sqrt{\lambda_j}$ \qquad $j=1,\dots,\min\{m,n\}$
\begin{equation*}
S=\begin{pmatrix}
\sigma_1 & & & 0 & \dots & 0\\
 & \ddots & &  \vdots &  &  \vdots\\
& & \sigma_m & 0 & \dots & 0\\
\end{pmatrix} \in \mathbb{R}^{m\times n}
\qquad m<n
\end{equation*}
\begin{equation*}
S=\begin{pmatrix}
\sigma_1 & & \\
 & \ddots & \\
& & \sigma_n \\
0 & \dots & 0\\
\vdots &  &  \vdots\\
0 & \dots & 0\\
\end{pmatrix} \in \mathbb{R}^{m\times n}
\qquad m>n
\end{equation*}
\item Bestimme $u_1,\dots,u_r$ aus $u_i=\frac{1}{\sigma_j}Av_j$ für alle $j=1,\dots,r$ (alle $\sigma_j\ne 0$)
\item Falls $r<m$ ergänze $u_1,\dots,u_r$ zu einer ONB, bzw. zu $U=\begin{pmatrix}
u_1 & \dots & u_m
\end{pmatrix}$
orthogonal.
\item $A=USV^\top$
\end{enumerate}

\subsection{Lineare Differentialgleichungen}
\subsubsection{Lösen einer linearen Differentialgleichung}
Gegeben: $y'(t) = \lambda y(t)$, mit $y(0) = c$\\
Lösung: $y(t) = ce^{\lambda t}$ mit $c \in \mathbb R$\\
\subsubsection{Lösen eines Systems linearer Differentialgleichungen}
Gegeben: $y'(t) = Ay$, mit $y_1(0) = c_1 , ..., y_n(0) = c_n$\\ 
Wobei $A \in \mathbb R^{nxn}, c_1, ..., c_n \in \mathbb R, y,y'\in\mathbb R^n$.\\
Lösung:\\
\begin{enumerate}
	\item Eigenwerte und Eigenvektoren von $A$ bestimmen.
	\item $\begin{pmatrix}
	x_1(t)\\
	\vdots\\
	x_n(t) \\
	\end{pmatrix}
	= c_1e^{\lambda_1t}v_1+...+c_ne^{\lambda_nt}v_n$.
	\item Anfangswerte einsetzen und Werte für $c_1$, bis $c_n$ bestimmen.
\end{enumerate}
\end{multicols*}
% Ende der Spalten


% Dokumentende
% ======================================================================
\end{document}